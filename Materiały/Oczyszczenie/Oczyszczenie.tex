%! TeX program = lualatex
\documentclass[10pt,a4paper]{instrukcja}

\fancyhead{}
\fancyfoot[C]{Imprimatur 		-- ks. Ireneusz Bakalarczyk}
\fancyfoot[L]{Nihil Obstat		-- Jakub Gajewski}
\fancyfoot[R]{Skład 			-- Michał Siemaszko}

\begin{document}

%%%%%%%%%%%%%%%%%%%%%%%%%%%%%%%%%%

\header{Oczyszczenie NMP}
%%%%%%%%%%%%%%%%%%%%%%%%%%%%%%%%%%
\section{Poświęcenie Gromnic i procesja}

\begin{itemize}
	\item Do prezbiterium wychodzimy normalnie, kapłan ubrany w białą kapę
	\item Po \textit{Dominus vobiscum} następuje 5 oracji po czym zasypanie,
	      pokropienie i okadzenie (potrzebny turyfer i ministrant z kropielnicą)
	      \footnote{\textbf{UWAGA}: może być tak, że celebrans pójdzie pokropić
		      świece wiernych}
	\item Świece rozdawane są ministrantom w sposób podobny do przyjmowania
	      Komunii Św.
	      \footnote{w trakcie rozdawania śpiewa się \textit{Pieśń Symeona}}
	\item Po zakończeniu rozdawania świec następuje ostatnia oracja
	\item Formuje się procesja z krzyżem i akolitami. Idą w niej wszyscy
	      ministranci, celebrans oraz  wierni. Wszyscy mają \underline{zapalone świece}.
	\item Idziemy dookoła kościoła i wracamy do prezbiterium
	\item Po powrocie celebrans zakłada biały ornat i wszyscy gaszą gromnice
\end{itemize}

\section{Zmiany na Mszy Świętej}

\begin{itemize}
	\item Brak modlitw u stopni ołtarza
	\item Msza z dnia bez perturbacji
	\item Zapalamy gromnice na Ewangelię i od \textit{Sanctus} do końca Kanonu.
	\item Zaleca się zrobienie lucenarium
\end{itemize}

\end{document}
