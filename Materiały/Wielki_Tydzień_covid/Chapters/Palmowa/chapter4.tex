\section{Dodatek}
\label{sec:dodatek_palm}

\subsection[Losy subdiakona krucyfera]{Losy subdiakona krucyfera \protect \footnote{Wymagany level -- obłuczony kleryk}}
\begin{itemize}
      \item subdiakon krucyfer ( \ding{63} ), ubrany w albę, cingulum i czerwoną
            tunicellę
      \item nie odbiera palmy od \ii
      \item po skończonej procesji idzie do kaplicy zimowej, gdzie ściąga westymenta
            subdiakona, zakłada komżę, bierze biret i przechodzi do chóru
\end{itemize}

\subsection{Procesja do Ewangelii - dla zaawansowanych}

\subsubsection*{\textbf{Celebrans z subdiakonem i ceremoniarzem }}
\begin{itemize}
      \item kiedy skończy się śpiew \textit{Graduału} i \textit{Traktusa}, \ii~ wraz
            z \ss~ i \cc1 podchodzą do ołtarza - przyklękają
      \item \ii~ wchodzi na najwyższy stopień i odmawia modlitwę \textit{Munda cor}
      \item \ss~ w tym czasie przenosi mszał
      \item \ss~ przechodzi ze Mszałem na stronę Ewangelii i pozostaje przy nim
            asystując \ii~ przy kartkach
      \item \cc1, \tt, \aa1, \aa2 \textbf{nie} ustawiają się jak na Mszy
            śpiewanej (zostają na swoich bazach) \footnote{Ceremoniał o.
                  Małaczyńskiego, str. 65, pkt. 47(dla rytu uroczystego): „Jeżeli Męka
                  Pańska nie jest śpiewana celebrans pod koniec traktusa odmawia
                  \textit{Munda cor} i czyta Mękę Pańską po stronie Ewangelii.
                  Równocześnie należy ją odczytać wiernym  po polsku."}
      \item \ii~ po przeczytaniu Pasji nie schodzi do sedilli tylko idzie na
            stronę lekcji i obraca się w kierunku śpiewanej pasji; \ss~ stoi na
            swoim stopniu (\textit{in plano}) po stronie lekcji odwrócony w
            stronę pasji, mogą wziąć palmy
\end{itemize}

\subsubsection*{\textbf{Diakon i śpiewacy}}
\begin{itemize}
      \item \dd, przy pomocy \cc2, zdejmuje dalmatykę i manipularz
      \item \dd, od \cc2, otrzymuje księgę z tekstem Pasji
      \item \cc2 prowadzi \dd~ do ambony, \eighthnote1 i \eighthnote2 czekają na
            miejscu; śpiew pasji wykonywany jest po stronie ewangelii przed
            balaskami
      \item wszystkie skłony podczas Pasji wykonujemy w stronę
            ewangeliarza/lekcjonarza (nie w stronę Mszału),
      \item po skończonej Pasji \dd~ i \cc2 przyklękają na środku, po czym \cc2
            prowadzi \dd~ do sedilli i pomaga mu założyć dalmatykę i manipularz
      \item \dd~ zajmuje swoje miejsce za \ii~ i przyklęka --
            \ii~ rozpoczyna \textit{Credo}
\end{itemize}