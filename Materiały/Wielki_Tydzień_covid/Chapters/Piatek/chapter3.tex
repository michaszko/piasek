\section{Uroczyste modły}

\begin{itemize}
      \item \zz~ przynosi z zakrystii czarną kapę
      \item \ii~ zakłada czarną kapę przy pomocy \cc1 i \cc2
      \item  w tym czasie \aa1 oraz \aa2 rozkładają na ołtarzu jeden obrus oraz
            ustawiają na środku pulpit z mszałem oraz kartką z wezwaniem
            modlitwy powszechnej za żydów, potem wracają na swoje miejsca przy
            kredencji
      \item  \ii~ wraz z \cc1 i \cc2 przytrzymującymi brzegi kapy podchodzi
            przed ołtarz, wykonują ukłon i wchodzą na stopnie
      \item  \ii~ całuje mensę ołtarzową i rozpoczyna śpiew modlitw
      \item  na wezwanie \textit{Flectamus genua} wszyscy klękają, na
            \textit{Levate} – wstają
      \item  po zakończeniu modlitw \ii~, \cc1 i \cc2 skłaniają głowy i wracają
            do sedilli krótką drogą, \cc1 i \cc2 trzymają kapę
      \item \tt1 zabiera mszał wraz ze stojakiem z ołtarza
      \item  \ii~ zdejmuje kapę, którą Z odnosi do zakrystii
\end{itemize}