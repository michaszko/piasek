\section{Szczegóły dotyczące głównie mszy bardziej uroczystych i solennych}

Na tych mszach często występują dodatkowe elementy, niezbyt często praktykowane,
dlatego warto będzie raz na zawsze ustalić odpowiednią wersję ich wykonania, we
wszystkich zdarzających się wariantach.\\

\noindent W miarę możliwości do dużych mszy wyznaczamy dwóch ceremoniarzy: \cc1
zwyczajnie asystuje \ii przy ołtarzu, a \cc2 prowadzi procesje i lucenarium,
ustawia dwójki do komunii,a także na bieżąco uzupełnia braki i
niedociągnięcia.\\

Kolejność okadzania (po \ii)
\begin{itemize}
	\item w solennej
	      \begin{itemize}
		      \item \dd~ okadza
		            \begin{itemize}
			            \item księży w chórze
			            \item kleryków w chórze
			            \item \ss~
		            \end{itemize}
		      \item \tt~ okadza
		            \begin{itemize}
			            \item \dd~
			            \item \cc~
			            \item \aa\aa~
			            \item ministrantów
			            \item lud
		            \end{itemize}
	      \end{itemize}
	\item w śpiewanej \tt~ okadza
	      \begin{itemize}
		      \item księży w chórze (każdego osobno)
		      \item kleryków (w chórze lub służących do Mszy)
		      \item \cc~ (jeśli nie jest klerykiem)
		      \item \aa\aa~
		      \item ministrantów
		      \item lud
	      \end{itemize}
\end{itemize}

