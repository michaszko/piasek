\section{Wystawienie Najświętszego Sakramentu}

\opis{Potrzebne: kapa, monstrancja, welon naramienny, \\ mikrofon bezprzewodowy}

\textbf{Zasady ogólne:}
\begin{itemize}
	\item Ministranci posługujący nie klękają podczas wystawienia na stopniu
	      ołtarza, ale na podłodze!
	\item Podczas wystawienia na ołtarzu nie powinny stać tablice ołtarzowe,
	      relikwie, kielich, mszał
	\item Jeśli jest piąty ministrant, który słucha Mszy w chórze, asystuje on
	      \ii~ razem z \cc~ podczas wystawienia. Przynosi on kapę do przebrania
	      \ii~, a także podtrzymuje brzegi kapy podczas błogosławieństwa.
	\item Jeśli nie ma piątego ministranta, kapę do przebrania przynosi \tt~, a
	      podczas błogosławieństwa brzeg kapy podtrzymuje \aa.
	\item \aa\aa~ w odpowiednich momentach podają \cc~, a także odbierają od
	      niego: książki z modlitwami, welon naramienny, inne potrzebne rzeczy.
	\item Podczas wystawienia nie błogosławi się kadzidła, nie całuje się dłoni
	      \ii, nie oddaje się nikomu rewerencji (pokłonów). W chórze nie
	      siadamy, lecz pozostajemy w pozycji klęczącej.
\end{itemize}

\textbf{Krok po kroku:}
\begin{itemize}
	\item Przed Mszą wyznacza się ministranta z chóru do pełnienia funkcji
	      asystenta przy wystawieniu. Jeśli jest tylko 4 ministrantów, kapę \ii~
	      przynosi \tt~, a podczas błogosławieństwa brzegi kapy podtrzymuje \aa.
	\item Asystent (jeśli go nie ma, to \tt~) – podczas modlitwy
	      \textit{Postcommunio} udaje się do bocznego ołtarza po kapę.
	\item Po odśpiewaniu \textit{Benedicamus Domino} \ii~ z \cc~ klęka przed
	      ołtarzem i udaje się do sedilli.
	\item Asystent (lub \tt~) trzyma kapę jak parawan przy sedilli, w tym czasie
	      \cc~ pomaga \ii~ zdjąć ornat i manipularz. Na znak \cc asystent
	      zakłada \ii~ kapę.
	\item \tt~ udaje się po kadzielnicę i przynosi ją.
	\item W tym czasie \aa\aa~ :
	      \begin{itemize}
		      \item kładą tablice ołtarzowe tak, aby nie przeszkadzały w
		            wystawieniu
		      \item jeśli \ii~ tego nie zrobił, rozkładają korporał z bursy obok
		            tabernakulum
		      \item ściągają z ołtarza relikwie
		      \item umieszczają na ołtarzu monstrancję po stronie Ewangelii,
		            otworem zwróconą w kierunku korporału
		      \item wracają na swoje miejsca
	      \end{itemize}
	\item Kiedy \aa\aa~ wypełnili swoje zadania, \cc~ daje znak \ii~.
	\item \ii~ w asyście \cc~ i asystenta (jeśli jest) udają się do ołtarza i
	      przyklękają. Asystent zajmuje miejsce po stronie Ewangelii, a \cc~ po
	      stronie Epistoły. \ii~ wchodzi po stopniach ołtarza do tabernakulum i
	      dokonuje wystawienia.
	\item Kiedy \ii~ wszedł na ostatni stopień, \cc~ daje znak wszystkim do
	      uklęknięcia na dwa kolana. {\boldmath{\cc}}~\textbf{ i asystent
		      klękają na podłodze, nie na stopniu ołtarza!}
	\item  Kiedy \ii~ dokona wystawienia i schodzi na dół, klęka na pierwszym
	      stopniu ołtarza. Razem z \cc~ i asystentem wykonują głęboki ukłon
	      przed Sanctissimum.
	\item Potem wstają, \tt~ podchodzi do \cc. Następuje zasypanie
	      kadzidła. Asystent podtrzymuje prawy brzeg kapy, \cc~ podaje łódkę. Nie
	      prosi się \ii~ o błogosławieństwo, nie całuje się łyżeczki.
	\item  Po zasypaniu \cc~ oddaje łódkę \tt~ i odbiera od niego trybularz.
	\item Po uklęknięciu \cc~ podaje \ii~ trybularz bez pocałunków.
	\item Po wykonanym głębokim ukłonie okadza się Najświętszy Sakrament. \cc~ i
	      asystent podtrzymują brzegi kapy.
	\item \cc~ klęcząc oddaje trybularz \tt~, który odchodzi na bok.
	\item \aa~ podaje klęczącemu \cc~ książkę z modlitwami z mikrofonem. \cc~ podaje ją \ii~.
	      Asystent umieszcza mikrofon przed \ii~.
	\item Brzegi kapy układa się elegancko na drugim stopniu ołtarza, tak żeby
	      zakrywały przestrzeń przed \ii. Nie powinny niechlujnie zwisać po
	      bokach!
	\item W razie potrzeby \cc~ i asystent pomagają \ii~ z książką i modlitwami.
	\item Po skończonych modlitwach, kiedy intonuje się \textit{Tantum ergo}, na
	      słowa \textit{Veneremur cernui} (\textit{Upadajmy wszyscy wraz})
	      wszyscy skłaniają się nisko.
	\item Potem \tt~ zbliża się do \cc~ i następują drugie zasypanie i okadzenie
	      Sanctissimum.
	% \item Po okadzeniu \aa~ podaje \cc~ welon naramienny.
	\item \ii~ sam wstaje i śpiewa oracje przed błogosławieństwem. (Jeśli
	      potrzeba, podaje się mu tekst)
	\item  Po oracjach \ii~ klęka. \aa~ podaje mu welon naramienny.
	\item \ii, \cc~ i asystent (lub \aa) wstępują na stopnie ołtarza. Podczas
	      błogosławieństwa Sanctissimum \cc~ i asystent klęcząc na najwyższym
	      stopniu ołtarza (twarzą do ciebie) podtrzymują brzegi kapy. Po błogosławieństwie
	      przyklękają na najwyższym stopniu razem z \ii~.
	\item \aa~ dzwoni dzwonkiem podczas błogosławieństwa.
	\item Po przyklęknięciu na najwyższym stopniu \ii~ ponownie klęka na
	      pierwszym stopniu ołtarza, a \cc~ i asystent klękają na podłodze.
	      Wykonują głęboki ukłon.
	\item \cc~ odbiera welon od \ii. \aa~ podchodzi do \cc, odbiera od niego
	      welon, składa go i kładzie na kredencji.
	\item Jeśli odmawia się \textit{Divinae Laudationes} (\textit{Niech będzie
		      Bóg uwielbiony...}), \cc~ podaje tekst \ii~.
	\item \ii~ ponownie wstępuje sam na najwyższy stopień i dokonuje schowania
	      Najświętszego Sakramentu.
	\item \aa~ dzwoni dzwonkiem na zamknięcie tabernakulum i wtedy \cc~ daje
	      wszystkim znak do powstania.
	\item Wyjście z prezbiterium jak zawsze po każdej Mszy Świętej.
\end{itemize}