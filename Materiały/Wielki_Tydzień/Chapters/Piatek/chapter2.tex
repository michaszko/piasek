\section{Czytania}

    \begin{itemize}
        \item wszyscy procesyjnie udają się do ołtarza i rozchodzą się na boki (bez skłonu)
        
        \begin{figure}[h]
        	\centering
            \includegraphics[scale=0.7]{Piatek/Wejscie.pdf}
        \end{figure}
        
        \item \ii~ wraz z ministrantami wykonują skłon, po czym \ii~ pada na twarz, a ministranci klękają na posadce i skłaniają głowy
        \item \ii~ wraz z \cc1 wstają i śpiewana jest modlitwa; ministranci prostują się
        \item po modlitwie wszyscy powstają, robią skłon i udają się na swoje miejsca (\ii, \aa1, \aa2 i \cc~ na sedile)
        \item \cc2 zabiera poduszkę ze stopni ołtarza
        \item \tt~ przynosi pulpit na środek prezbiterium 
        \item wszyscy siadają razem z \ii
        \item śpiewana jest lekcja, po niej następuje responsorium i modlitwa
        \item na \textit{Oremus} wszyscy wstają, na \textit{Flectamus genua} - klękają, na \textit{Levate} - wstają, po skończonej modlitwie - siadają
        \item następuje kolejne czytanie i responsorium
        \item podczas responsorium \tt wynosi pulpit z prezbiterium, a \cc2 przenosi pulpit na stronę Ewangelii i kładzie go na posadce
        \item po responsorium \ii~ odmawia modlitwę przed stopniami ołtarza po czym udaje się do pulpitu i czyta Męke Pańską
        \item ministranci po jej zakończeniu nie mówią \textit{Laus tibi, Christe}
    \end{itemize}