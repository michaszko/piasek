\section{Obrzęd Komunii Św.}

\begin{itemize}
   \item \ii~ zmienia szaty - asystuje \cc1
   \item \ii~ zanosi korporał na ołtarz w towarzystwie \cc1 i \cc2
   \item \aa1 i \aa2 ustawiają się przed stopniami ołtarza i wraz z \oo~, \cc1,
         \cc2 i \ii~ procesyjnie udają się to ołtarza przechowywania
   \item podczas powrotu procesji do ołtarza, \cc1 i \cc2 uderzają kołatką, a
         wszyscy ministranci klęczą
   \item po dojściu do stopni ołatarza \oo~ odchodzi na bok, \aa1 i \aa2
         odkładają akolitki na ołtarzu, a \cc1, \cc2 i \ii~ podchodzą do ołarza

         \begin{figure}[h]
            \centering
            \includegraphics[scale=0.6]{Piatek/NSakrament.pdf}
         \end{figure}

   \item przy ołtarzu \cc2 odbiera welon, po czym przyklęka i idzie na swoje
         miejsce
   \item wszyscy ministranci wstają; w odpowiednim momencie za \ii~ odmawiają \textit{Pater noster}
   \item na znak \cc1 ministranci ustawiają się parami na posadzce do przyjęcia
         Komunii i na kolejny znak - klękają
   \item w tym czasie \aa1 (z obrusem komunijnym) i \aa2 klękają twarzami do
         siebie na najwyższym stopniu ołtarza
   \item \cc1 rozpoczyna \textit{Confiteor}, do którego dołączają się pozostali
   \item po \textit{Indulgentiam...} \aa1 i \aa2 rozciągają obrus komunijny
         między sobą
   \item pierwsi do Komunii przystępują ministranci od pateny i świeczki
         sanctusowej; zaraz po jej przyjęciu asystują \ii~
   \item do Komunii ministranci przystępują w następujący sposób (więcej:
         \hyperref[komunia]{patrz Dodatek}):

         \begin{enumerate}[leftmargin=1cm]
            \item para przyklęka na posadzce równocześnie z poprzedzającą ją
                  parą
            \item po wejściu na stopnie ołtarza, klęka na najwyższym stopniu
            \item przyjąwszy Komunię, przyklęka w miejscu i udaje się na swoje
                  miejsce
         \end{enumerate}

   \item po przyjęciu Komunii św. przez ministrantów, \aa1 i \aa2 wstają i udają
         się z obrusem komunijnym do balasek; razem z nimi przechodzą \cc1 i
         \cc2
   \item po skończonej Komunii wiernych, \aa1 i \aa2 składają obrus komunijny i
         odkładają na kredensji i przy niej pozostają
   \item wszyscy wstają, gdy zostanie schowany Najświętszy Sakrament i pozostają
         w tej pozycji
   \item gdy \ii~ oczyści patenę, zabiera ją \aa1; korporał nie jest chowany!
   \item \ii~ śpiewa trzy modlitwy po komuni; po nich \cc2 zamyka i ściąga z
         ołtarza mszał
   \item \ii~ udaje się do sedilii i zmienia szaty - asystuje \cc1
\end{itemize}

