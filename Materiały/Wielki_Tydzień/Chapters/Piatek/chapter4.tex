\section{Adoracja Krzyża}

\begin{itemize}
    \item po modlitwach \ii~ z \cc1 i \cc2 udają się krótką drogą do sedilii,
          gdzie \ii~ zdejmuje kapę - asystują \cc1 i \cc2
    \item formuje się procesja w kolejności:

          \begin{enumerate}\centering
              \item[] (stopnie ołtarza)
              \item[] \cc2~~~\ii~~~~\cc1
              \item[] \aa2~~~\aa1
          \end{enumerate}

    \item uczestniczący w procesji skłaniają się w stronę ołtarza i przechodzą
          do zakrystii krótką drogą

    \item z zakrystii rusza procesja w kolejności:

          \begin{enumerate}\centering
              \item[] \cc2~~~\cc1
              \item[] \aa2~~~\ii~~~~\aa1
              \item[]
              \item[] gdzie \aa~ trzymają świece, \ii~ - krzyż
          \end{enumerate}

    \item procesja długą drogą zmierza stopni ołtarza; po dojściu staje na
          posadzce po stronie epistoły

          \begin{figure}[h]
              \centering
              \includegraphics[scale=0.7]{Piatek/Obraz.pdf}
          \end{figure}

    \item \ii~ odsłaniając krzyż, śpiewa: \textit{Ecce lignum crucis}, na co
          wszyscy odpowiadając: \textit{Venite, adoremus} i klękają (ten schemat
          powtarza się w sumie 3-krotnie)
    \item \aa1 i \aa2 zostawiają akolitki po obu stronach krzyża, po czym
          odbierają od \ii~ krzyż
    \item \aa2 przynosi poduszkę na stopnie ołtarza; \aa1 i \aa2 kładą na nią
          krzyż
    \item \cc1 i \aa2 schodzą razem z \ii~ do sedilii i wszyscy ściągają obuwie

    \item adoracja następuje w kolejności:

          \begin{enumerate}\centering
              \item[] \ii~
              \item[] \cc1
              \item[] \aa2
              \item[] ministranci (dopóki nie przyjdą \aa1 i \aa2)
              \item[] \aa1
              \item[] \aa2
              \item[] reszta ministrantów
          \end{enumerate}

    \item krzyż adorowany będzie przez ministrantów pojedynczo w sposób
          następujący:

          \begin{enumerate}[leftmargin=1cm]
              \item pierwsze przyklęknięcie ma miejsce przy balaskach
              \item drugie - w połowie prezbiterium
              \item trzecie - przy stopniach ołatarza
              \item następuje klęknięcie i pocałunek krzyża
              \item po wstaniu, ministrant wstaje, robi krok w bok, przyklęka z
                    ministrantem, który właśnie doszedł do krzyża  i udaje się
                    na swoje miejsce
              \item wszystkie przyklęknięcia wykonuje się \textbf{w tym samym
                        czasie}, co ministrant przed nami!
          \end{enumerate}

          \begin{figure}[h]
              \centering
              \includegraphics[scale=0.6]{Piatek/Adoracja.pdf}
          \end{figure}

    \item \cc1 i \aa2 po dokananiu adoracji, odbierają od \aa1 i \aa2 krzyż i go
          trzymają
    \item \aa1 i \aa2 po dokonaniu adoracji i nałożeniu butów, klękają po obu
          stronach krzyża twarzą do siebie na środkowym stopniu ołtarza przy
          akolitkach
    \item po skończeniu adoracji przez ministrantów, \cc1 i \aa2 zanoszą krzyż
          przed prezbiterium w towarzystwie \aa1 i \aa2 ze świecami
    \item \aa1 i \aa2 stwaiają świece po bokach krzyża i odbierają od \cc1 i
          \aa2 krzyż i go trzymają
    \item po skończonej adoracji \cc1 i \aa2 odbierają od \aa1 i \aa2 krzyż i
          zanoszą go do stopni ołtarza
    \item tam dołącza do nich \ii~ i wszyscy trzej wchodzą po stopniach i \ii~
          umieszcza krzyż na tabernakulum
    \item w międzyczasie \aa1 i \aa2 biorą swoje świece i stawiają je na ołtarzu
\end{itemize}
