\section{Poświęcenie palm i procesja}

\begin{itemize}
	\item w tej części \dd~ wykonuje pocałunki \ii~ standardowo jak podczas mszy
	\item podczas ubierania (na {\color{red} czerwono}) celebransa asystuje \cc1,
	      kolejno diakona \aa1 oraz subdiakona \aa2
	\item ustawiamy się w szyku procesyjnym. \dd~ i \ss~ przez cały czas
	      procesji i poświęcenia, jeżeli jest to możliwe, trzymają kapę.
	      Wychodzimy przez główne wyjście na zewnątrz. Przy przechodzeniu przez
	      oś ołtarza polowego, przyklękają wszyscy oprócz: \ding{63}, \aa1, \aa2
	      oraz \ii
	\item schola śpiewa podczas procesji \textit{Hosanna filio David} na
	      przemian z Psalmem 117 \footnote{Zob. \textit{Liber Usualis} z 1961 :
		      Ritus servandus in celebratione Missae in Cantu ad 1.}
	\item po dojściu na miejsce święcenia palm, \cc1 odbiera nakrycia głowy,
	      następuje  oddanie referencji ołtarzowi przez \dd~ i \ss~ oraz \ii~
	      (\dd~ i \ss~ przyklękają, \ii~ głęboko się skłania), \ii~ całuje
	      ołtarz następnie ustawiamy się w sposób przedstawiony na Rys.
	      \ref{fig:przyjscie}

	      \begin{figure}[h]
		      \centering
		      \includegraphics[width=0.8\linewidth]{Palmowa/PalmyNadOdra.pdf}
		      \caption{Ustawienie po przyjściu do ołtarza}
		      \label{fig:przyjscie}
	      \end{figure}

	\item \ii~ cicho odczytuje antyfonę \textit{Hosanna filio David}, tak jak
	      podczas Mszy, a następnie śpiewa \textit{Dominus vobiscum}, zwrócony w
	      kierunku księgi
	\item po \textit{Amen} następuje zasypanie
	\item \aa1 podaje wodę \cc1 a ten podaje ją \dd
	\item następuje pokropienie, \aa1 odbiera wodę od \dd
	\item następuję okadzenie
	\item palmę dla \ii~ \dd~ kładzie na ołtarzu
	\item następuje rozdawanie palm (\cc2 podaje palmy, \cc1 zawiaduje ruchem),
	      odbieranie palm od \ii~ i \dd~ z pocałunkiem, asysta ustawia się
	      dwójkami jak do komunii w kolejności: \dd~ i \ss~, klerycy,
	      ministranci, lud \footnote{Diakon może rozdawać palmy wtedy i tylko
	      wtedy ($\iff$) gdy jest ustanowiony w stopniu święceń prezbitera}
	\item po rozdaniu palm \ii~ i \dd~ myją ręce, pomagają mu w tym \aa\aa
	\item następuje odśpiewanie ewangelii, nie przenosi się mszału
	\item do ewangelii ustawiamy się jak podczas mszy, gdy już dojdziemy na
	      miejsce jak na Rys. \ref{fig:ewangelia}

	      \begin{figure}[h!]
		      \centering
		      \includegraphics[width=0.8\linewidth]{Palmowa/PalmyNadOdra2.pdf}
		      \caption{Ustawienie podczas Ewangelii}
		      \label{fig:ewangelia}
	      \end{figure}

	\item po odśpiewaniu ewangelii \ii~ całuje ewangeliarz oraz jest okadzany
	      przez \dd
	\item kantorzy (\spiew1) oraz mikrofoniarz udają się najkrótszą możliwą drogą
	      do pierwszych drzwi kościoła. Przymykają je lekko, ale obserwują, czy
	      procesja już się zbliża. Dwóch lub jeden kantor (\spiew2) pozostaje w
	      procesji z ministrantami
	\item po powrocie do pozycji z Rys. \ref{fig:przyjscie} następuje zasypanie
	\item \dd~ i \ss~ stają w rzędzie za \ii
	\item \dd~ staje obok rzędu i śpiewa \textit{Procedeamus in pace}
	\item kantor (\spiew2) oraz mikrofon wraz z ludem śpiewa \textit{In nomine
		      Christi. Amen}, a następnie rozpocznie jedno z poniższych:

	      \begin{itemize}
		      \item śpiew jednej z antyfon procesyjnych z \textit{Liber
			            Usualis}
		      \item \textit{Christus vincit}
		      \item pieśń po polsku do Chrystusa Króla
	      \end{itemize}

	\item po oddaniu rewerencji ołtarzowi \cc1 oddaje nakrycia głowy
	\item ustawiamy się w szyku procesyjnym, \spiew2 zajmują w procesji miejsce
	      zaraz za \ding{63}
	\item udajemy się w kierunku bocznych drzwi po schodach za pomnikiem biskupa
	      Kominka
	\item $K1$ wewnątrz kościoła obserwują delikatnie, czy procesja nadchodzi
	\item kiedy procesja dochodzi do drzwi kościoła, \ding{63} z \aa\aa~ stają przodem
	      do drzwi, a ministranci rozchodzą się na boki tworząc szpaler tak aby
	      \ii~ znajdował się nieopodal \ding{63}, jak na Rys. \ref{fig:procesja_palm}

	      \medskip

	      \begin{figure}[h]
		      \centering
		      \includegraphics[width=0.26\linewidth]{Palmowa/PalmyNadOdra3.pdf}
		      \caption{Procesja przy drzwiach kościoła}
		      \label{fig:procesja_palm}
	      \end{figure}

	\item \spiew1 (w środku) stojąc zwróceni w kierunku drzwi śpiewają:
	      \textit{Gloria Laus et honor} etc.
	\item \spiew2, ministranci i wierni powtarzają werset: \textit{Gloria Laus}
	\item \spiew1 śpiewają poszczególne zwrotki hymnu, po każdej zwrotce
	      \spiew2 i reszta odśpiewują refren: \textit{Gloria, Laus}
	\item po piątej zwrotce i refrenie \ding{63} głośno i widocznie uderza nóżką
	      krzyża procesyjnego w drzwi, a \spiew1 i \cc2 szeroko otwierają drzwi
	\item \cc2 wprowadza \ding{63} i resztę procesji do wnętrza kościoła. W tym
	      czasie \spiew1 i \spiew2 łączą się w jedną grupę i jak najszybciej zaczynają
	      śpiew: \textit{Ingrediente Domino}, podany w
	      \textit{Liber Usualis}
	\item \dd, \ss, \ii~ oraz \cc~ po przyklęknięciu na środku udają się
	      bezpośrednio do księgi po stronie epistoły, reszta asysty i chór
	      przyklękają po kolei na środku. Wszyscy zajmują swoje miejsca i
	      odkładają palmy. \aa\aa~ odkładają świece a \ding{63} krzyż na stojak
	\item modlitwa na zakończenie procesji od Mszału ustawionego przy ołtarzu po
	      stronie Epistoły, \dd~ i \ss~ trzymają brzegi kapy (jak przy poświęceniu)
	\item po skończonej oracji, \ii~ i \dd~ oraz \ss~ podchodzą do sedilli i
	      przebierają się (w {\color{violet}fiolety}). Przebieranie przebiega w
	      następującej kolejności:

	      \begin{itemize}
		      \item \cc1 zabiera kapę przekazuje \zz
		      \item \dd~ i \ss~ ściągają dalmatyki z pomocą \aa1 oraz \aa2
		      \item \cc\cc~ ubierają w ornat \ii, \zz\zz~ zabierają w tym
		            czasie dalmatyki i manipularze czerwone
		      \item \aa1 zakłada tunicelę \ss, \aa2 dalmatykę \dd
		      \item \cc1 sprawdza czy \dd~ i \ss~ mają odpowiednie szaty
	      \end{itemize}

	\item w czasie przebierania schola śpiewa \textit{Introit}
\end{itemize}