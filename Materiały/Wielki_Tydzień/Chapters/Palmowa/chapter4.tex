\section{Dodatek}
\subsection{Losy subdiakona krucyfera}
\begin{itemize}
  \item subdiakon krucyfer ( \ding{63} ), ubrany w albę, cingulum i czerwoną
        tunicellę
  \item nie odbiera palmy od \ii.
  \item po skończonej procesji idzie do kaplicy zimowej, gdzie ściąga westymenta
        subdiakona, zakłada komżę, bierze biret i przechodzi do chóru.
\end{itemize}

%\newpage

\subsection{Prosecja do Ewangelii - dla zaawansowanych}
\label{E}
\subsubsection*{\textbf{Celebrans z subdiakonem i ceremoniarzem }}
\begin{itemize}
  \item kiedy skończy się śpiew Graduału i Traktusa, \ii~ wraz z
        \ss~ i \cc1 podchodzą do ołtarza - przyklękają
  \item \ii~ wchodzi na najwyższy stopień i odmawia modlitwę \textit{Munda cor}
  \item \ss~ w tym czasie przenosi mszał,
\end{itemize}

\subsubsection*{\textbf{Diakon}}
\begin{itemize}
  \item przy pomocy \cc2 zdejmuje dalmatykę i manipularz
  \item od \cc2 otrzymuje księgę z tekstem Pasji, a następnie zajmuje
        miejsce pomiędzy dwoma śpiewakami - \ss1 i \ss2
\end{itemize}

\subsubsection*{\textbf{Ale jak to wygląda?}}

\begin{enumerate}\centering
  \item[] \ii
  \item[] (stopnie ołtarza)
  \item[] \ss
  \item[]
  \item[]
  \item[] \ss1~~~\dd~~~\ss2
  \item[] \cc2
\end{enumerate}

% \newpage

\begin{itemize}
  \item na znak \cc2 wszyscy oprócz \ii~ przyklękają
  \item \ss~ przechodzi ze Mszałem na stronę Ewangelii i pozostaje przy
        nim asystując \ii~ przy kartkach
  \item \cc1, \tt, \aa1, \aa2
        \textbf{nie} ustawiają się jak na Mszy śpiewanej (zostają na swoich
        'bazach') i skłaniają się do krzyża ołtarzowego razem z \ii~
        \footnote{Ceremoniał o. Małaczyńskiego, str. 65, pkt. 47(dla rytu
          uroczystego): „Jeżeli Męka Pańska nie jest śpiewana celebrans pod koniec
          traktusa odmawia „Munda cor” i czyta Mękę Pańską po stronie Ewangelii.
          Równocześnie należy ją odczytać wiernym  po polsku."}
  \item[]
  \item \cc2 prowadzi \ss1, \dd~, \ss2 do
        ambony – gdzie następuje śpiew Pasji po polsku
  \item wszystkie skłony podczas Pasji wykonujemy w stronę
        ewangeliarza/lekcjonarza (nie w stronę Mszału),
  \item[]
  \item po odśpiewanej Pasji po polsku \ii~ staje przed środkiem
        ołtarza, a \ss~ za nim na swoim stopniu.
  \item \dd~, \ss1, \ss2 i \cc2
        przyklękają na środku, po czym \cc2 prowadzi \dd~ do
        sedilli i pomaga mu założyć dalmatykę i manipularz
  \item \dd~ zajmuje swoje miejsce za \ii~ i przyklęka --
        \ii~ rozpoczyna \textit{Credo}
\end{itemize}

%  \section{Do poprawy} \begin{itemize} \item {\color{red}Przypomnieć
%    ministrantom, żeby na Ewangelie skłaniali głowy \textbf{do Ewangeliarza}, a
%    nie do ołtarza.} \item \sout{Jeśliby dobrze poukładać palmy na stoliku, tak
%    żeby całość była stabilna, to można by się pokusić o 'wjechanie' nim na
%    środek jak już przejdzie cała procesja.} \end{itemize}