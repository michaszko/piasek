\section{Dodatek}

\subsection{Przebieg poświęcenia wody}
\label{sec:woda}
\begin{enumerate}
      \item \textit{Dominus Vobiscum} i Oracja 1.
      \item \textit{Dominus Vobiscum} i Oracja 2.
      \item Prefacja do słów \textit{Sumat unigeniti tui gratiam de Spiritu
                  Sancto}
      \item Celebrans kreśli znak krzyża na wodzie
      \item Wyciera ręce
      \item Kontynuuje modlitwę
      \item Na \textit{Sit haec} dotyka palcem wody
      \item Kreśli znaki krzyża na \dots \textit{super te ferebatur}
      \item Wylewa wodę na cztery strony świata
      \item Po zmianie głosu na \textit{recto tono} trzy razy dmucha do wody na
            kształt krzyża
      \item W tym czasie ceremoniarz podaje Paschał
      \item Trzy razy wkłada Paschał do wody i śpiewa
      \item \ii~ trzy razy dmucha do wody w kształcie litery {\Large ${\Psi}$}
      \item Wyciągamy Paschał z wody
      \item Po wyjęciu paschału nabiera się wody do pokropienia
      \item Następnie \cc1 przynosi do chrzcielnicy tackę z olejami świętymi i
            wręcza \ii~ (z pocałunkiem) odpowiednie ampułki. \ii, wypowiadając
            słowa przepisane w księdze po kolei:
            \begin{itemize}
                  \item wlewa olej katechumenów
                  \item wlewa krzyżmo
                  \item wlewa oba
            \end{itemize}
      \item Następnie miesza wodę ręką lub przy pomocy łyżeczki.
      \item Później myje i wyciera ręce \footnote{W razie potrzeby wykorzystuje
                  sól, watę i miękisz chleba, które później należy spalić. Wodę
                  z tej ablucji wlewa
                  się do sacrarium.}
      \item Zasypanie kadzidła i okadzenie chrzcielnicy
\end{enumerate}

\subsection{Bierzmowanie}
\label{sec:bierz}
\begin{itemize}
      \item księgę z modlitwami trzyma \cc1, a \cc2 podtrzymuje kapę jeśli
            trzeba; \aa1 i \aa2 stoją z boku przy kredencji
      \item \ii~ odmawia modlitwę zwracając się do kandydatów \textit{Spiritus
                  Sanctus superveniat\dots}
      \item \ii~ czyni znak Krzyża Świetego mówiąc \textit{Adjutorium
                  nostrum...} a następnie kontynuowany jest krótki dialog
      \item \ii~ wyciąga ręce nad przystępującymi do bierzmowania i mówi
            \textit{Oremus. Omnipotens sempiterne Deus,...}
      \item przystępujący do sakramentu klęka przed \ii. Świadek kładzie rękę na
            prawym ramieniu bierzmowanego.
      \item \aa1 podchodzi do \ii~ z Krzyżmem Świętem
      \item \ii~ kładzie prawą rękę na głowie bierzmowanego i palcem umoczonym w
            Krzyżmie Świętym robi znak krzyż a na czole, mówiąc \textit{Signo te
                  signo...}
      \item \ii~ uderza lekko w policzek bierzmowanego, mówiąc \textit{Pax tecum}
      \item kandydat wraca na swoje wcześniejsze miejsce i stoi
      \item \aa1 odbiera Krzyżmo, a \aa2 podchodzi z czymś do wytarcia palców
            \footnote{najpewniej wata}; później wracają do kredencji
      \item gdy \ii~ namaści wszystkich kandydatów, śpiewa się antyfonę
            \textit{Confirma hoc, Deus\dots}
      \item powtarza się antyfonę, a następnie \ii~ zwrócony do ołtarza śpiewa
            \textit{Ostende nobis, Domine,...}, a następnie \textit{Oremus.
                  Deus, qui Apostolis tuis\dots}
      \item \ii~ mówi \textit{Ecce sic benedicetur omnis homo, qui timet Dominum}
      \item na końcu \ii~ błogosławi bierzmowanych
            \textit{Bene + dicat vos Dominus ex Sion...}
\end{itemize}