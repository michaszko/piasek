\section{Msza Św. i procesja do Bożego Grobu}

\subsection{Msza Św.}

\begin{itemize}
	\item po przebraniu się od udajemy się procesyjnie do Ołtarza i rozpoczyna
	      się Msza św.
	\item wyznaczone osoby udają się do chrzcielnicy i do kaplicy św. Iwa i tam
	      sprzątają
	\item zmiany na Mszy Św.
	      \begin{itemize}
		      \item nie ma modlitw u stopni Ołtarza \footnote{nie ma \textit{Confiteor},
			            od razu \textit{Aufer a nobis}}
		      \item podczas pierwszego okadzenia \aa1 nie musi Mszału zabierać z Ołtarza,
		            bo ten jest na stoliku, ale trzeba go po okadzeniu ustawić na ołtarz
		      \item \aa\aa~ na \textit{Gloria} dzwonią dzwonkami; w tym czasie w
		            sygnaturkę uderza \tt~ i \zz; dzwoni sie do skończenia
		            \textit{Gloria}	przez \ii
		      \item po Graduale \ii~ intonuje uroczyste, trzykrotne alleluja
		            (po stroni lekcji)
		      \item nie ma \textit{Agnus Dei}
		      \item przed komunią (po przyjęciu komunii przez \ii) \cc1 zaczyna
		            \textit{Confiteor}
		      \item po puryfikacji kielicha trzeba go ściągnąć z ołtarza -- dlatego
		            \aa2 do przeniesienia welony ubiera białe rękawiczki i ubraniu
		            kielicha przez \ii~ znosi go do kredencji
		      \item po skończonej puryfikacji następują \textit{Laudesy}:
		            \begin{itemize}
			            \item chór śpiewa pierwszą Antyfonę (\textit{Laudate Dominum})
			            \item \ii~ staje po stornie epistoły (towarzyszy \cc1) i
			                  czyta śpiewaną Antyfonę
			            \item kiedy chór zakończy śpiew Antyfony, \ii~ intonuje
			                  Antyfonę do Kantyku Zachariasza (\textit{Et valde
				                  mane}), a dalej śpiewa chór
			            \item następnie \ii~ czyta, a chór śpiewa Kantyk
			                  Zachariasza (\textit{Benedictus Dominus})
			                  \footnote{żegnamy się kiedy chór śpiewa
				                  \textit{Benedictus}, czyli na początku Kantyku}
			            \item kiedy \ii~ skończy czytać następuje normalne
			                  zasypanie i okadzenie ołtarza oraz usługujących i
			                  wiernych (tak jak podczas ofiarowania z
			                  wyłączeniem okadzenia darów)
			            \item \textit{Gloria Patri} w Kantyku scholi wolno
			                  zaśpiewać dopiero wtedy, gdy \tt1 okadzi już
			                  wszystkich
		            \end{itemize}
		      \item po skończonych \textit{Laudesach} kiedy chór skończy śpiewać
		            \ii~ śpiewa \textit{Postcommunio}
		      \item po \textit{Ite missa est, alleluia, alleluia} i błogosławieństwie nie
		            ma ostatniej ewangelii.
	      \end{itemize}
\end{itemize}

\subsection{Wystawienie Najświętszego Sakramentu}

\begin{itemize}
	\item schola zaraz po \textit{Ite missa est, alleluia, alleluia} \ii~
	      zaczyna śpiewać \textit{Cum Rex Glori\ae}
	\item \ii~ schodzi (z przyklęknięciem przy stopniach ołtarza) do sedilli i
	      przebiera biały ornat na białą kapę, którą przynosi \zz
	\item w międzyczasie \aa\aa~ ściągają z ołtarza tablice i relikwie, a
	      zostawiają na nim monstrancje
	\item \cc1, \ii~ i \cc2 udają się do ołtarza i następuje wystawienie
	      Najświętszego Sakramentu, zasypanie i okadzenie (\cc1 i \cc2 są ciągle
	      po obu strona \ii)
	\item \ii~ intonuje \textit{Gloria Tibi}, a schola śpiewa dalej
	\item potem dialog \ii~ z ludem i \textit{Oracja}
	\item w międzyczasie \ding{63} bierze krzyż z założoną już na niego stułą i
	      stoi przy kredencji (nie klęka kiedy inni klękają)
	\item po skończonym śpiewie \ding{63} podchodzi do \cc1, klęka i przekazuje
	      mu krzyż, a ten z koleji przekazuje go \ii~ ciągle go podtrzymując
		  \footnote{najlepiej ja krzyż jest trzymany przez \ii~ i \cc1/\cc2 przez cały czas}
	\item następnie \ii~ śpiewa \textit{Surrexit Dominus de sepulcro} podnosząc
	      przy tym krzyż; lud odpowiada \textit{Qui pro nobis pependit in ligno,
		      alleluia}
	\item to podnoszenie następuje w sumie trzykrotnie, za każdym razem podnosi
	      się głos przy śpiewaniu
	\item po tym następuje dialog z ludem i \textit{Oracja}
	\item po niej \ding{63} odbiera krzyż i odnosi go na miejsce; później wraca
	      do kredencji
	\item następnie organista śpiewa uroczyste \textit{Te Deum}
	\item po skończonym śpiewie \ii~ błogosławi lud NS i chowa go
	\item na koniec \ii~ intonuje \textit{Regina Caeli} i po jej zaśpiewaniu
	      schodzimy procesyjnie do zakrystii
\end{itemize}