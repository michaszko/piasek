\section{Msza Święta}

\begin{itemize*}
      \item Po modlitwie, \ii~ przebiera się przy sedilli w ornat i zakłada
            manipularz. Jeśli msza jest solenna, \dd~ (i \ss) ubiera(ją)
            manipularz(e). Dalej msza toczy się jak zwykle, z następującymi
            wyjątkami i wskazaniami:
            \begin{itemize*}
                  \item Odmawiamy całe modlitwy u stopni ołtarza.
                  \item Na mszy solennej \ii~ odczytuje lekcję i ewangelię
                  cicho.
                  \item W trakcie śpiewu traktusu na słowa \textit{Adiuva nos,
                        Deus salutaris noster} klękamy.
                  \item Uwaga! \ii~ nie przyklęka podczas odczytywania
                        traktusu --  robi to razem ze wszystkimi podczas śpiewu
                        scholi, na środku na pierwszym stopniu.
                  \item Po modlitwie po komunii jest modlitwa nad ludem. W mszy
                        solennej \dd, po \textit{Oremus} \ii~, obraca się do ludu
                        i śpiewa \textit{Humiliate capita vestra Deo}.
                  \item Na koniec zamiast \textit{Ite, missa est} śpiewa się
                        \textit{Benedicamus Domino} przodem do ołtarza.
                  \item Ważne: Chór klęczy na oba kolana:
                        \begin{itemize*}
                              \item podczas kolekty,
                              \item od \textit{Sanctus} do \textit{Pax Domini},
                              \item podczas modlitwy po komunii i modlitwy nad
                              ludem.
                        \end{itemize*}

            \end{itemize*}

\end{itemize*}