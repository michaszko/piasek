\section{Liturgia słowa}

\begin{itemize}
	\item lud i wszyscy ministranci -- poza \ii~ i funkcyjnymi -- powinni mieć
	      świece zapalone od początku \footnote{Jeżeli się święci ogień to od
		      poświęcenia ognia}
	\item po przybyciu do prezbiterium wszyscy przyklękamy przy stopniach
	      ołtarza i rozchodzimy się na swoje miejsca:
	      \begin{itemize}
		      \item \ii~ idzie po stopniach do ołtarza, całuje ołtarz i udaje
		            się do mszału
		      \item \dd~ i \ss~ pomagają \ii~ wejść, a potem stają na swoich
		            stopniach (za księdzem, tak jak podczas introitu)
		      \item \cc1 idzie do mszału (będzie asystował \ii~ przy czytaniu
		            proroctw)
		      \item \cc2 idzie do chóru
		      \item \aa1 i \aa2 idą do kredencji
		      \item pierwszy kantor idzie od razu do ambonki i zaczyna śpiewać
		            jak tylko \ii~ znajdzie się przy ołtarzu
		      \item reszta idzie do chóru
	      \end{itemize}
	\item \ii~ czyta po cichu tekst proroctwa (ewentualnie także kantyku)
	      \footnote{Towarzyszy mu \cc1 ze zniczem (gdy jest zbyt ciemno)} i, po
	      skłonie do krzyża (razem z \dd~ i \ss), krótką drogą udaje się do
	      sedilli, razem z nim \dd~ i \ss
	\item na znak \cc1~ \ii, \dd, \ss~ wstają i razem, krótką drogą udają się do Mszału
	\item \ii~ śpiewa \textit{Oremus} ze skłonem\footnote{Nie ma
		      \textit{Flectamus genua} oraz \textit{Levate}}, a potem śpiewa
	      orację
	\item po skończonej oracji czyta po cichu proroctwo i cykl się zamyka
	\item \aa\aa~ podczas czytania przez \ii~ lekcji stoją przy kredencji, a
	      kiedy \ii~ schodzi do sedilli udają się tam razem z nim
	\item chór siedzi podczas lekcji a stoi podczas oracji
	\item kantorzy śpiewający
	      \begin{itemize}
		      \item jeśli proroctwo poprzedza traktus to podchodzą do ambonki w
		            trakcie jego trwania
		      \item jeśli proroctwa nie poprzedza traktus to podchodzą do
		            ambonki pod koniec poprzedzającego proroctwa i stoją z
		            poprzednim kantorem, który odchodzi od ambonki po oracji
		            (stoją obok siebie podczas trwania oracji)
	      \end{itemize}
\end{itemize}