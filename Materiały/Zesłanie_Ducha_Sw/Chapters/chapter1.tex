\section{Przygotowanie do obrzędów}

\subsection{Zakrystia}
\begin{itemize}
	\item Na obrzędy wstępne chrztu:
	      \begin{itemize}
		      \item dla \ii: alba simplex, cingulum, \textcolor{violet}{kapa i
			            stuła fioletowe}, biret
		      \item olej katechumenów
		      \item wata
		      \item sól
		      \item księgę z modlitwami (\textit{Rytuał Wrocławski})
	      \end{itemize}
	\item Na czuwanie:
	      \begin{itemize}
		      \item dla \ii: alba simplex, cingulum, \textcolor{violet}{ornat,
			            stuła i manipularz fioletowe}, biret
		      \item dla \dd: alba simplex, cingulum, \textcolor{violet}{plikata,
			            stuła i manipularz fioletowe}, biret
		      \item dla \ss: alba simplex, cingulum, \textcolor{violet}{plikata
			            i manipularz fioletowe}, biret
		      \item komże simplex dla ministrantów
	      \end{itemize}
	\item Na Mszę
	      \begin{itemize}
		      \item dla \ii: alba koronka, cingulum, \textcolor{red}{ornat,
			            stuła i manipularz czerwone}, biret
		      \item dla \dd: alba koronka, cingulum, \textcolor{red}{dalmatyka,
			            stuła i manipularz czerwone}, biret
		      \item dla \ss: alba koronka, cingulum, \textcolor{red}{tunicella
			            i manipularz czerwone}, biret
		      \item komże koronkowe dla ministrantów
	      \end{itemize}
\end{itemize}

\subsection{Baptysterium}
\begin{itemize}
	\item chrzcielnica umyta, napełniona świeżą wodą, przyozdobiona
	      białym materiałem, zielenią i kwiatami \footnote{Zgodnie z polskim
		      zwyczajem można obsypać kwiatami posadzkę w baptysterium}
	\item \textcolor{black!50}{biała kapa i stuła}
	\item dodatkowy stojak na Paschał
	\item księga poświęcenia wody (np. OHS) + stojak
	\item ręcznik dla księdza do wytarcia rąk
	\item pusty kociołek na wodę i kropidło
	\item rytuał lub inną księgę zawierającą obrzędy chrztu
	\item naczynie z solą (błogosławioną lub do pobłogosławienia)
	\item naczynia z olejem katechumenów i Krzyżmem do namaszczenia katechumena
	\item naczynie lub muszlę do udzielania chrztu
	\item czysty ręcznik do otarcia głowy ochrzczonego
	\item biała szata oraz świeca chrzcielna dla każdego katechumena
	      (przynoszone przez rodziców chrzestnych)
\end{itemize}

\subsection{Kaplica bierzmowania (św. Iwo)}
\begin{itemize}
	\item ołtarz nakryty obrusem, na predelli krzyż i  4. lichtarze ze świecami
	\item siedzenie bez oparcia dla celebransa - na środku suppedaneum, przy
	      ołtarzu;
	\item rytuał, pontyfikał lub inną księgę z obrzędami bierzmowania;
	\item krzyżmo w naczyniu z watą;
	\item tacę z watą lub odpowiednie opaski płócienne do przewiązania w miejscu
	      namaszczenia dla bierzmowanych;
	\item tacę z chlebem i solą oraz naczynie z wodą i misę do obmycia rąk
	      celebransa;
	\item gremiał płócienny dla ochrony szat przed zabrudzeniem olejami (może
	      być np. stary humerał)
\end{itemize}

\subsection{Kredensja ołtarzowa}
\begin{itemize}
	\item Ewangeliarz
	\item kielich, ampułki, cyborium z komunikantami
	\item akolitki
	\item świeczka dla \cc1 do oświetlania Mszału
	\item relikwiarze
	\item wszystkie dzwonki jakie mamy
	\item \textcolor{violet}{3 poduszki fioletowe}
	\item mszaliki z łacińskim tekstem dla kontrolowania przebiegu liturgii
	      % \item 6 świec ołtarzowych
\end{itemize}

\subsection{Ołtarz główny}
\begin{itemize}
	\item antypendium \textcolor{black!50}{białe}, na nim założony
	      {\color{violet}fioletowy} materiał, który można łatwo usunąć.
	\item 12 białych świec na złotych świecznikach
	\item kwiaty
	\item odkryty krzyż złoty
	\item mszał otwarty na proroctwach
	\item obrus rozwinięty
	\item koronka ołtarzowa zawinięta
\end{itemize}

\subsection{Inne}
\begin{itemize}
	\item w pierwszym rzędzie ławek: klęcznik lub ławka z zarezerwowanym
	      miejscem dla katechumena i rodziców chrzestnych
	\item mały klęcznik okryty na biało dla katechumena do przyjęcia Pierwszej
	      Komunii Świętej
	\item ambonka z tekstami proroctw w miejscu, z którego jest
	      normalnie śpiewana lekcja na Mszy
	\item stojak na mikrofon w kruchcie na obrzędy wstępne
\end{itemize}

\hrulefill

\subsection{Wprowadzenie katechumena}
\begin{itemize}
	\item kilkanaście minut przed celebracją do kruchty kościoła udają się \ii
	      \footnote{W \textcolor{violet}{fioletowej kapie} jeśli chrzczony jest
		      dorosły}, \dd, \ss, \cc1 i \cc2
	\item następuje wprowadzenie katechumena do kościoła (patrz
	      \nameref{sec:chrzest})\footnote{Zaczynamy na końcu kościoła, potem
		      idziemy do ołtarza NOM, a potem bokiem do chrzcielnicy}
	\item po zakończeniu obrządku wszyscy udają się do zakrystii
	\item \ii~ zmienia \textcolor{violet}{fioletową kapę} na
	      \textcolor{violet}{fioletowy ornat i manipularz}, a \dd~ ubiera
	      \textcolor{violet}{fioletowy manipularz}
\end{itemize}

\hrulefill


