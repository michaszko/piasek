\begin{center}
	\vspace*{2cm}

	\header{{\Huge Obrzędy Wigilii \\ Zesłania Ducha Świętego}\\
		\bigskip{\large pre 1955}}

	\vspace{2cm}

	{\large \textbf{Użyte oznaczenia:}} \\

	\vspace{0.05\textwidth}

	{\large
		\begin{table}[!h]
			\large
			\hspace{6cm}
			\begin{tabular}{r c l}
				\ii                  & -- & celebrans           \smallskip \\
				\dd                  & -- & diakon              \smallskip \\
				\ss                  & -- & subdiakon           \smallskip \\
				\cc                  & -- & ceremoniarz         \smallskip \\
				\aa                  & -- & akolita             \smallskip \\
				\tt                  & -- & turyfer             \smallskip \\
				\ding{63}            & -- & krzyż procesyjny    \smallskip \\
				% \oo                  & -- & ombrelino           \smallskip\\
				\zz                  & -- & zakrystianin        \smallskip \\
				% \bb                  & -- & ministrant księgi   \smallskip\\
				\spiew~ \eighthnote~ & -- & śpiewający          \smallskip \\
				% \kolatki             & -- & kołatki             \smallskip\\
				\paschal             & -- & ministrant Paschału \smallskip \\
			\end{tabular}
		\end{table}
	}

	\vspace{1cm}

	\begin{figure}[!htbp]
		\centering
		\includegraphics[width=0.3\linewidth]{Figures/logo.png}
	\end{figure}

	\footer{by Michał Siemaszko, Aleksander Glapiak, Maciej Rumin; maj 2020}
\end{center}

