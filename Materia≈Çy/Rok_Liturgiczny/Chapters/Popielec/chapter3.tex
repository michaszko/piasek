\section{Poświęcenie popiołu}
\begin{itemize*}
	\item \ii~ udaje się do mszału i odczytuje modlitwy. Towarzyszy mu \dd~ i
	      \ss, jeśli msza jest solenna, oraz \cc1. W trakcie modlitw powinien
	      podejść \tt~ z rozpalonym kadzidłem w trybularzu i łódką oraz jeden
	      z \aa~ z wodą święconą.
	\item Po odczytaniu modlitw kolejność czynności jest następująca:
	      \begin{itemize*}
		      \item zasypanie,
		      \item pokropienie popiołu wodą święconą,
		      \item okadzenie popiołu.
	      \end{itemize*}
	\item Następnie \ii~ posypuje głowę popiołem na stopniach ołtarza w
	      kolejności zgodnej z precedencją:
	      \begin{itemize*}
		      \item sobie, klękając na najwyższym stopniu (jeśli jest inny ksiądz,
		            to on posypuje głowę \ii),
		      \item (\dd~ i \ss),
		      \item księża,
		      \item klerycy,
		      \item ministranci.
	      \end{itemize*}
	\item Jeśli \ii~ posypuje sobie sam głowę popiołem, nie wypowiada w tym
	      momencie żadnych słów. Posypując głowę innym \ii~ wypowiada słowa:
	      \textit{Memento homo quia pulvis es et in pulverem reverteris}.
	\item Stajemy w parach, jak do komunii.
	\item Potem \ii~ udaje się, by posypać głowy ludu.
	\item Gdy kończy się obrzęd posypania głów popiołem ludowi, \aa\aa~
	      przygotowują przy kredencji miskę, dzbanek z wodą, ręcznik i mydło. Po
	      umyciu rąk, \ii~ wraca do ołtarza i odmawia modlitwę na zakończenie
	      obrzędu.
\end{itemize*}