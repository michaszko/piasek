%! TeX program = lualatex
%! TEX options = -synctex=1 -interaction=nonstopmode -file-line-error --shell-escape "%DOC%"
\documentclass[10pt,a5paper]{mszalik}

\begin{document}

\thispagestyle{empty}

\begin{center}
	\vspace*{0.5cm}


	\bfseries\scshape
	\huge REQUIEM

	\vfill

	\pgfornament[width=7cm]{88}

	\vfill

	\begin{figure}[h]
		\centering
		\includegraphics[width=0.5\linewidth]{logo.png}
	\end{figure}

	% \vfill

	{\large Duszpasterstwo Wiernych Tradycji Łacińskiej \\ w Archidiecezji Wrocławskiej}

	\bigskip

	{\Large \textbf{2021}}


\end{center}

% \afterpage{\null\thispagestyle{empty}\newpage}

% \newpage
% \thispagestyle{empty}

\vspace*{2cm}
"Popiół jest symbolem znikomości świata oraz pokory i pokuty. Poddając się
obrzędowi posypania głów popiołem, uznajemy publicznie naszą grzeszność i wyrażamy
chęć poprawy. Na mocy zasług Kościoła otrzymujemy pomoc Bożą do gorliwego odbycia
Wielkiego Postu"

\bigskip

\hfill \textit{Mszał Rzymski, Wydawnictwo Pallotinum, Poznań, 1963}

\vspace*{\fill}

\hrulefill

{\footnotesize Niniejszy mszalik przygotowany jest w celu wyjaśnienia obrzędów i
	edukacji liturgicznej wiernych. Teksty liturgiczne wraz z tłumaczeniami oraz
	grafiki pochodzą z:

	\begin{itemize}[leftmargin=*]
		\item \textit{Mszału Rzymskiego z dodaniem nabożeństw nieszpornych}
		      autorstwa O. Gaspara Lefebvre (Poznań 1932)
		\item \textit{Mszału Rzymskiego} autorstwa benedyktynów tynieckich
		      (Poznań 1963)
		\item \textit{Missale Romanum} wyd. Typis Polyglotis Vaticanis (1927)
		      oraz Pustet (1899)
		\item \textit{Cantus Selecti} (1957)
		\item \textit{Liber Usualis} (1961)
	\end{itemize}

	W niektórych przypadkach wykorzystano cyfrową wersję powyższych tekstów
	opublikowaną na portalu \texttt{divinumofficium.com}.

	\medskip

	Cytaty z homilii papieża Benedykta XVI za \texttt{opoka.org.pl}.

	\medskip

	Teksty czytań opatrzone komentarzem i wyjaśnieniem zaczerpnięto z
	\textit{Biblii Tysiąclecia Online} (Poznań 2003). Zostały one wykorzystane
	na mocy prawa cytatu. (art. 29. Ustawy o prawie autorskim i prawach
	pokrewnych) Prawa autorskie do nich posiada wyd. Pallotinum. Tekst Biblii
	Tysiąclecia jest oficjalnym tekstem Kościoła w Polsce zatwierdzonym do
	stosowania w liturgii.

	\bigskip

	opracowanie: Maciej Rumin\\
	nadzór merytoryczny ks. dr Ireneusz Bakalarczyk\\
	skład i łamanie: Michał Siemaszko\\

	email: tradycja@archidiecezja.wroc.pl
}
\hrule

%	\afterpage{\null\thispagestyle{empty}\newpage}
\newpage

%%%%%%%%%%%%%%%%%%%%%%%%%%%%%%%%%%%%%%%%%%%%%%%%%%%%%%%%%%%%%%%%%%%%%%%%%%

\newpage

\fancyhead[C]{MSZA ŚWIĘTA}

\begin{figure}[h]
	\centering
	\includegraphics[width=\linewidth]{Figures/popielec.jpg}
\end{figure}

\section{MSZA ŚWIĘTA}
 % {\station{Stacja u św. Sabiny}}

\oremus{Introit (4 Esdr. 2, 34 et 35, Ps. 64, 2-3.)}{

Réquiem ætérnam dona eis, Dómine: et lux perpétua lúceat eis. \\

Te decet hymnus, Deus, in Sion, et tibi reddétur votum in Ierúsalem: exáudi
oratiónem meam, ad te omnis caro véniet. \\\\

Réquiem ætérnam dona eis, Dómine: et lux perpétua lúceat eis. \medskip

\vv Glória Patri, et Fílio, et Spirítui Sancto. \\\\
\rr Sicut erat in princípio, et nunc, et semper, et in sǽcula sæculórum. Amen.}{

Wieczny odpoczynek racz im dać, Panie, a światłość wiekuista niechaj im
świeci. \\

Ciebie, Boże, przystoi wielbić hymnem na Syjonie; Tobie dopełnią ślubów w
Jeruzalem; wysłuchaj mojej modlitwy: do Ciebie przyjdzie każdy człowiek. \\

Wieczny odpoczynek racz im dać, Panie, a światłość wiekuista niechaj im
świeci. \medskip

\vv Chwała Ojcu, i Synowi i Duchowi Świętemu.\\
\rr Jak była na początku, teraz i zawsze i na wieki wieków. Amen.}


\oremus{Kolekta}{
	Fidélium, Deus, ómnium Cónditor et Redémptor: animábus famulórum famularúmque
	tuárum remissiónem cunctórum tríbue peccatórum; ut indulgéntiam, quam semper
	optavérunt, piis supplicatiónibus consequántur: Qui vivis et regnas cum Deo
	Patre, in unitate Spiritus Sancti Deus, per omnia saecula saeculorum.  
	}{
	Boże, Stwórco i Odkupicielu wszystkich wiernych, udziel duszom zmarłych
	sług i służebnic Swoich odpuszczenia wszystkich grzechów: niech przez pobożne
	modły nasze dostąpią miłosierdzia, którego zawsze pragnęły. Który żyjesz i
	królujesz z Bogiem Ojcem w jedności Ducha Świętego Bóg przez wszystkie wieki
	wieków. Amen.}

\proroctwo{{Lekcja (1 Cor 15:51-57)}}

% \rubric{Kościół przypomina nam przede wszystkim mocne wezwanie, jakie prorok Joel
% 	skierował do narodu Izraelskiego (...). Podkreślić należy wyrażenie «całym swym sercem»,
	% które oznacza, że nawrócenie pochodzi z głębi naszych myśli i uczuć, leży u podstaw
	% naszych decyzji, wyborów i działań, w geście całkowitej i radykalnej wolności.
	% Ale czy możliwy jest ten powrót do Boga? Tak, jest możliwy, bo istnieje siła,
	% która nie pochodzi z naszego serca, ale która uwalnia się w sercu samego Boga.
	% Jest to siła Jego miłosierdzia. --- Benedykt XVI, 13.02.2013}

\oremuss{ Fratres: Ecce, mystérium vobis dico: Omnes quidem resurgámus, sed non
	omnes immutábimur. In moménto, in ictu óculi, in novíssima tuba: canet enim
	tuba, et mórtui resúrgent incorrúpti: et nos immutábimur. Opórtet enim
	corruptíbile hoc induere incorruptiónem: et mortále hoc indúere
	immortalitátem. Cum autem mortále hoc indúerit immortalitátem, tunc fiet
	sermo, qui scriptus est: Absórpta est mors in victória. Ubi est, mors,
	victória tua? Ubi est, mors, stímulus tuus? Stímulus autem mortis peccátum
	est: virtus vero peccáti lex. Deo autem grátias, qui dedit nobis victóriam
	per Dóminum nostrum Iesum Christum.
	}{
	Bracia: Oto wam powiadam tajemnicę: Nie wszyscy wprawdzie zaśniemy, ale
	wszyscy będziemy przemienieni. Nagle, w oka mgnieniu, na odgłos trąby
	ostatecznej, albowiem zabrzmi trąba i umarli powstaną nieskażeni, a my
	będziemy przemienieni. Bo to, co skazitelne, musi przyoblec się w
	nieskazitelność i to, co śmiertelne, przyoblec się w nieśmiertelność. A gdy
	to, co śmiertelne, przyoblecze się w nieśmiertelność, tedy wypełni się słowo,
	które jest napisane: Pochłonęło śmierć zwycięstwo. Gdzież jest zwycięstwo
	twe, o śmierci? Gdzie jest, o śmierci, oścień twój? A ościeniem śmierci jest
	grzech, siłą zaś grzechu jest Prawo. A Bogu dzięki, że nam dał zwycięstwo
	przez Pana naszego Jezusa Chrystusa.}

\proroctwo{Graduał (4 Esdr. 2, 34 et 35, V. Ps. 111, 7.)}

\oremuss{Réquiem ætérnam dona eis, Dómine: et lux perpétua lúceat eis.

	\vv In memória ætérna erit iustus: ab auditióne mala non timébit.}
{Wieczny odpoczynek racz im dać, Panie, a światłość wiekuista niechaj im świeci.

	\vv W wiecznej pamięci będzie sprawiedliwy: nie będzie się lękał smutnej
	nowiny.}

\proroctwo{Traktus} 

\oremuss{
Absólve, Dómine, ánimas ómnium fidélium ab omni vínculo delictórum. \\

\vv Et grátia tua illis succurrénte, mereántur evádere iudícium ultiónis.

\vv Et lucis ætérnæ beatitúdine pérfrui.
}{
Uwolnij, Panie, dusze wszystkich wiernych zmarłych od wszelkich więzów grzechowych.

\vv A przy pomocy Twojej łaski niechaj unikną strasznego wyroku zatracenia.

\vv I niech zażywają szczęścia wiecznej światłości.}

\newpage

\proroctwo{Sekwencja}

\oremuss{
Dies iræ, dies illa,\\
Solvet sæclum in favilla:\\
Teste David cum Sibylla.\\

Quantus tremor est futurus,\\
Quando Iudex est venturus,\\
Cuncta stricte discussurus!\\

Tuba, mirum spargens sonum\\
Per sepulchra regionum,\\
Coget omnes ante thronum.\\

Mors stupebit, et natura,\\
Cum resurget creatura,\\
Iudicanti responsura.\\

Liber scriptus proferetur,\\
In quo totum continetur,\\
Unde mundus iudicetur.\\

Iudex ergo cum sedebit,\\
Quidquid latet, apparebit:\\
Nil inultum remanebit.\\

Quid sum miser tunc dicturus?\\
Quem patronum rogaturus,\\
Cum vix iustus sit securus?\\

Rex tremendæ maiestatis,\\
Qui salvandos salvas gratis,\\
Salva me, fons pietatis.\\

Recordare, Iesu pie,\\
Quod sum causa tuæ viæ:\\
Ne me perdas illa die.\\

Quærens me, sedisti lassus:\\
Redemisti Crucem passus:\\
Tantus labor non sit cassus.\\

Iuste Iudex ultionis,\\
Donum fac remissionis\\
Ante diem rationis.\\

Ingemisco, tamquam reus:\\
Culpa rubet vultus meus:\\
Supplicanti parce, Deus.\\

Qui Mariam absolvisti,\\
Et latronem exaudisti,\\
Mihi quoque spem dedisti.\\

Preces meæ non sunt dignæ:\\
Sed tu bonus fac benigne,\\
Ne perenni cremer igne.\\

Inter oves locum præsta,\\
Et ab hædis me sequestra,\\
Statuens in parte dextra.\\

Confutatis maledictis,\\
Flammis acribus addictis,\\
Voca me cum benedictis.\\

Oro supplex et acclinis,\\
Cor contritum quasi cinis:\\
Gere curam mei finis.\\

Lacrimosa dies illa,\\
Qua resurget ex favílla\\
Iudicandus homo reus:\\
Huic ergo parce, Deus:\\

Pie Iesu Domine,\\
Dona eis requiem. Amen.\\
}{
Dzień ów gniewu się nachyla,\\
gdy w proch wieki zmiecie chwila,\\
świadkiem Dawid i Sybilla.\\

Będzie strach tam, będzie drżenie,\\
przyjdzie sędzia sądzić ziemię\\
a roztrząsać wszystko wiernie.\\

Trąba wyda głos dokoła,\\
gdzie kto gnił w mogilnych dołach,\\
wszystkich do stóp tronu zwoła.\\

Śmierć struchleje, wszelkie ciało\\
gdy powstanie, jak leżało,\\
by przed Sędzią głos zabrało.\\

Zwój ksiąg będzie rozwiniony\\
zapisanych z każdej strony,\\
z których ma być świat sądzony.\\

Sędzia zasię gdy rozkaże,\\
co ukryte, to ukaże,\\
a bezkarnie nic nie zmaże.\\

Cóż mam, nędzny, odpowiadać,\\
jakiego obrońcę badać,\\
gdy nawet dobremu biada.\\

Królu strasznej wielmożności,\\
co zbawiasz nas mimo złości,\\
wybaw mnie, źródło litości.\\

Pomnij, Jezu, miłościwy,\\
jakoś dla mnie czynił dziwy,\\
nie gub mnie w tym dniu straszliwym.\\

Dla mnieś się utrudził, Panie,\\
krzyża mękę cierpał za mnie,\\
miałby trud ten przepaść marnie?\\

Prawy Sędzio odemszczenia,\\
daj nam łaskę odpuszczenia\\
jeszcze przed dniem rozliczenia.\\

Od zbrodniarza jęczę srożej,\\
czoło ogniem winy gorze,\\
proszącego oszczędź, Boże.\\

Ty, co Marii odpuściłeś,\\
łotra z krzyża pocieszyłeś,\\
nadzieję w serce włożyłeś.\\

Prośby me tego niegodne,\\
ale ty wejrzyj łagodnie,\\
bym nie płonął wiecznym ogniem.\\

Postaw mnie w owiec zagrodzie,\\
od kozłów sprośnych mnie oddziel,\\
staw po prawicy w swym grodzie.\\

Pomieszawszy z przeklętymi,\\
ogniem srogim objętymi,\\
wywołaj mnie ze świętymi.\\

W kornej błagam Cię postawie,\\
z sercem startym na proch prawie,\\
uczyń zadość mojej sprawie.\\

Dzień nieszczęsny się nachyla,\\
gdy rozpęknie się mogiła,\\
wstanie na sąd człowiek grzeszny,\\
Ty go oszczędź, Boże wieczny.\\

Dobry Jezu a nasz Panie,\\
daj im swoje spoczywanie. Amen. \\
}

\proroctwo{Ewangelia (Ioann 5:25-29)}

% \rubric{Jezus odnosi się do trzech podstawowych praktyk Prawa Mojżeszowego:
% 	jałmużna, modlitwa i post; są one również tradycyjnymi wskazówkami dla
% 	wielkopostnego wędrowania, by odpowiedzieć na zaproszenie do «powrotu do
% 	Boga całym sercem». Jezus jednak podkreśla, że tym, co decyduje o
% 	autentyczności wszelkich gestów religijnych jest jakość i prawdziwość
% 	relacji z Bogiem. Dlatego właśnie Jezus demaskuje religijną hipokryzję
% 	jako zachowanie, które promuje zewnętrzność, przyjmując postawy,
% 	które zabiegają o poklask i uznanie. --- Benedykt XVI, 13.02.2013}

\oremuss{In illo tempore: Dixit Jesus turbis Judaeorum: Amen, amen dico vobis:
quia venit hora, et nunc est, quando murtui audient vocem Filii Dei: et qui
audierint, vivent. Sicut enim Pater habet vitam in semetipso, sic dedit et
Filio habere vitam in semetipso: et potestatem dedit ei judicium facere, quia
Filius hominis est. Nolite mirari hoc, quia venit hora, in qua omnes, qui in
monumentis sunt, audient vocem Filii Dei: et procedent, qui bona fecerunt, in
resurrectionem vitae: qui vero mala egerunt, in resurrectionem judicii.}
{Onego czasu: Rzekł Jezus rzeszom żydowskim: Zaprawdę, zaprawdę powiadam wam,
	że zbliża się godzina, a nawet już nadeszłą, gdy umarli usłyszą głos Syna
	Bożego, a który usłyszą, żyć będą. Jako bowiem Ojciec ma życie sam w sobie,
	tak dał i Synowi, aby miał życie w sobie samym. I dał władzę czynić sąd, bo
	Synem Człowieczym jest. Nie dziwcie się temu, bo nadchodzi godzina, w której
	wszyscy, co są w grobach, usłyszą głos Syna Bożego. I ci, którzy dobrze
	czynili, wyjdą na zmartwychwstanie życia, a którzy źle czynili, na
zmartwychwstanie sądu. }

\oremus{Offertorium (Ps 29:2-3)}{ Domine Jesu Christe, Rex gloriae, libera
	animas omnium fidelium defunctorum de poenis inferni et de profundo lacu:
	libera eas de ore leonis, ne absorbeat eas tartarus, ne candant in obscurum:
	sed signifer sanctus Michaël repraesentet eas in lucem sanctam: Quam olim
	Abrahae promisisti, et semini ejus. \\

\vv Hostias et preces tibi, Domine, laudis offerimus: tu suscipe pro animabus
illis, quarum hodie memoriam facimus: fac eas, Domine, de morte transire ad
vitam. Quam olim Abrahae promisisti, et semini ejus.}
{Panie Jezu Chryste, Królu chwały, zachowaj dusze wszystkich wiernych zmarłych
	od kar piekielnych i głębokiej czeluści. Wybaw je z lwiej paszczęki, niech
	ich nie pochłonie piekło, niech nie wpadają w ciemności. Lecz chorąży święty
	Michał niech je stawi w światłości świętej, którą niegdyś obiecałeś
	Abrahamowi i jego potomstwu.

\vv Składamy Ci, Panie, ofiary i modły pochwalne, a Ty je przyjmij za te dusze,
które dzisiaj wspominamy. Spraw, o Panie, niech przejdą ze śmierci do życia,
które niegdyś obiecałeś Abrahamowi i jego potomstwu.}

\oremus{Sekreta}{Hóstias, quaesumus, Dómine, quas tibi pro animábus famulórum
famularúmque tuárum offérimus, propitiátus inténde: ut, quibus fídei christiánæ
méritum contulísti, dones et praemium. Per Dominum nostrum Iesum Christum,
Filium tuum: qui tecum vivit et regnat in unitate Spiritus Sancti Deus, per
omnia saecula saeculorum.}
{Panie, wejrzyj łaskawie na ofiary, które Ci składamy za dusze sług i służebnic
	Twoich, i obdarz nagrodą tych, którym dałeś zasługę wiary chrześcijańskiej.
	Przez Pana naszego Jezusa Chrystusa, Syna Twojego, który z Tobą żyje i
	króluje, w jedności Ducha Świętego, Bóg przez wszystkie wieki wieków. Amen.}

\medskip

\rubric{Prefacja Communis.}

\newpage

\oremus{Communio (4 Esdr. 2, 35 et 34)}{
Lux aeterna luceat eis, Domine: Cum Sanctis tuis in aeternum: quia pius es.\\

\vv Requiem aeternam dona eis, Domine: et lux perpetua luceat eis. Cum Sanctis
tuis in aeternum: quia pius es.}
{Światłość wiekuista niechaj im świeci, o Panie, wśród Świętych Twoich na
	wieki, bo jesteś pełen dobroci.

\vv Wieczny odpoczynek racz im dać, Panie, a światłość wiekuista niechaj im świeci,
wśród świętych Twoich na wieki, bo jesteś pełen dobroci.}


\oremus{Modlitwa po Komunii {\color{red}JEST ZŁA - ZMIENIĆ}}{Percépta nobis,
		Dómine, præbeant sacraménta subsídium: ut tibi grata sint nostra ieiúnia,
		et nobis profíciant ad medélam.}{Niech nas wspomaga, Panie, przyjęty
		Sakrament, aby nasze posty były miłe Tobie, a nam przyniosły wyleczenie.}

\bigskip

\vfill

\centerline{\pgfornament[width=3cm]{106}}

% \newpage

\proroctwo{Antyfona}
%
\begin{center}
	Ave, Regína cælórum,\\
	Ave, Dómina Angelórum:\\
	Salve radix, salve porta,\\
	Ex qua mundo lux est orta:\\
	Gaude, Virgo gloriósa,\\
	Super omnes speciósa,\\
	Vale, o valde decóra,\\
	Et pro nobis Christum exóra.
\end{center}

\vfill

\centerline{\pgfornament[width=7cm]{82}}

\vfill


\end{document}







