%! TeX program = lualatex
%! TEX options = -synctex=1 -interaction=nonstopmode -file-line-error --shell-escape "%DOC%"
\documentclass[10pt,a5paper]{mszalik}

\begin{document}

\thispagestyle{empty}

\begin{center}
	\vspace*{0.5cm}


	\bfseries\scshape
	\huge ŚRODA POPIELCOWA

	\vfill

	\pgfornament[width=7cm]{88}

	\vfill

	\begin{figure}[h]
		\centering
		\includegraphics[width=0.3\linewidth]{logo.jpg}
	\end{figure}

	\vfill

	{\large Duszpasterstwo Wiernych Tradycji Łacińskiej \\ w Archidiecezji Wrocławskiej}

	\bigskip

	{\Large \textbf{2020}}


\end{center}

%	\afterpage{\null\thispagestyle{empty}\newpage}

\newpage
\thispagestyle{empty}

\vspace*{2cm}
"Oto misterium Pięćdziesiątnicy: Duch Święty oświeca ducha ludzkiego, a
objawiając ukrzyżowanego i zmartwychwstałego Chrystusa, wskazuje, na jakiej
drodze możemy upodobnić się do Niego, czyli być «wyrazem i narzędziem miłości,
która od Niego emanuje» (Deus caritas est, 33). Razem z Maryją, tak jak w chwili
swych narodzin, Kościół modli się dzisiaj: «Przyjdź, Duchu Święty! Przyjdź,
Duchu Święty, napełnij serca swoich wiernych i zapal w nich ogień swojej
miłości!» Amen."

\bigskip

\hfill \textit{Benedykt XVI, Homilia  w uroczystość Zesłania Ducha Świętego 2006}

\vspace*{\fill}

\hrulefill

{\footnotesize Niniejszy mszalik przygotowany jest w celu wyjaśnienia obrzędów i
	edukacji liturgicznej wiernych. Teksty liturgiczne wraz z tłumaczeniami oraz
	grafiki pochodzą z:

	\begin{itemize}[leftmargin=*]
		\item \textit{Mszału Rzymskiego z dodaniem nabożeństw nieszpornych}
		      autorstwa O. Gaspara Lefebvre (Poznań 1932)
		\item \textit{Mszału Rzymskiego} autorstwa benedyktynów tynieckich
		      (Poznań 1963)
		\item \textit{Missale Romanum} wyd. Typis Polyglotis Vaticanis (1927)
		      oraz Pustet (1899)
	\end{itemize}

	W niektórych przypadkach wykorzystano cyfrową wersję powyższych tekstów
	opublikowaną na portalu \url{divinumofficium.com}.

	\medskip

	Krótki cytat z homilli Benedykta XVI pochodzi z \textit{L'Osservatore
		Romano} (8/2006)

	\medskip

	Teksty czytań opatrzone komentarzem i wyjaśnieniem zaczerpnięto z
	\textit{Biblii Tysiąclecia Online} (Poznań 2003). Zostały one wykorzystane
	na mocy prawa cytatu. (art. 29. Ustawy o prawie autorskim i prawach
	pokrewnych) Prawa autorskie do nich posiada wyd. Pallotinum. Tekst Biblii
	Tysiąclecia jest oficjalnym tekstem Kościoła w Polsce zatwierdzonym do
	stosowania w liturgii.

	\bigskip

	opracowanie: M. Rumin, Ł. Wolański, M. Skarupski\\
	nadzór merytoryczny ks. dr Ireneusz Bakalarczyk\\
	skład i łamanie: Michał Siemaszko\\

	email: tradycja@archidiecezja.wroc.pl
}
\hrule

%	\afterpage{\null\thispagestyle{empty}\newpage}
\newpage

%%%%%%%%%%%%%%%%%%%%%%%%%%%%%%%%%%%%%%%%%%%%%%%%%%%%%%%%%%%%%%%%%%%%%%%%%%

\newpage

\fancyhead[CE,CO]{ŚRODA POPIELCOWA}

\begin{figure}[h]
	\centering
	\includegraphics[width=\linewidth]{Figures/popielec.jpg}
\end{figure}

\section{ŚRODA POPIELCOWA}
%  {\station{Stacja u św. Piotra}}

\rubric{Krótki opis co się będzie działo}

\proroctwo{Pieśń na wejście}

\gregorioscore{Melody/PD}

\newpage

\oremus{Introit (Sap 11:24 11:25; 11:27, Ps 56:2)}{ Miseréris ómnium, Dómine, et nihil
	odísti eórum quæ fecísti, dissímulans peccáta hóminum propter pæniténtiam et
	parcens illis: quia tu es Dóminus, Deus noster\\

	Miserére mei, Deus, miserére mei: quóniam in te confídit ánima mea. \medskip

	\vv Glória Patri, et Fílio, et Spirítui Sancto. \\
	\rr Sicut erat in princípio, et nunc, et semper, et in sǽcula sæculórum. Amen.}{
	Wszystkim okazujesz miłosierdzie, Panie, i żadnego ze swych stworzeń nie masz w
	nienawiści. Darowujesz grzechy ludziom, aby się nawrócili, i przebaczasz im:
	albowiem Ty jesteś Panem, Bogiem naszym.

	Zmiłuj się nade mną, Boże, zmiłuj się nade mną, bo Tobie ufa dusza moja.\medskip

	\vv Chwała Ojcu, i Synowi i Duchowi Świętemu.\\
	\rr Jak była na początku, teraz i zawsze i na wieki wieków. Amen.
}

\oremus{Kolekta}{Præsta, Dómine, fidélibus tuis: ut ieiuniórum veneránda
	sollémnia, et cóngrua pietáte suscípiant, et secúra devotióne
	percúrrant.}{Udziel, Panie, wiernym Twoim łaski, aby rozpoczęli ten czcigodny i
	uroczysty post z należną pobożnością i przebyli go służąc Tobie w pokoju.}

\proroctwo{{Epistoła (Jl 2:12-19 )}}

\rubric{Słowami proroka Joela Kościół wzywa nas do postu i wewnętrznego
	nawrócenia. Pokuta, którą głosił Prorok, miała charakter społeczny. Podobnie
	Wielki Post jest okresem wspólnego postu i wspólnej modlitwy całego Kościoła.}

\oremuss{Hæc dicit Dóminus: Convertímini ad me in toto corde vestro, in ieiúnio,
	et in fletu, et in planctu. Et scíndite corda vestra, et non vestiménta vestra,
	et convertímini ad Dóminum, Deum vestrum: quia benígnus et miséricors est,
	pátiens, et multæ misericórdiæ, et præstábilis super malítia. Quis scit, si
	convertátur, et ignóscat, et relínquat post se benedictiónem, sacrifícium et
	libámen Dómino, Deo vestro? Cánite tuba in Sion, sanctificáte ieiúnium, vocáte
	coetum, congregáte pópulum, sanctificáte ecclésiam, coadunáte senes, congregáte
	parvulos et sugéntes úbera: egrediátur sponsus de cubíli suo, et sponsa de
	thálamo suo. Inter vestíbulum et altare plorábunt sacerdótes minístri Dómini, et
	dicent: Parce, Dómine, parce pópulo tuo: et ne des hereditátem tuam in
	oppróbrium, ut dominéntur eis natiónes. Quare dicunt in pópulis: Ubi est Deus
	eórum? Zelátus est Dóminus terram suam, et pepércit pópulo suo. Et respóndit
	Dóminus, et dixit populo suo: Ecce, ego mittam vobis fruméntum et vinum et
	óleum, et replebímini eis: et non dabo vos ultra oppróbrium in géntibus: dicit
	Dóminus omnípotens.}{To mówi Pan: Nawróćcie się do mnie całym sercem waszym
	przez post, płacz i żal. I rozdzierajcie serca wasze, a nie wasze szaty. I
	nawróćcie się do Pana Boga waszego, bo łaskawy jest i litościwy, cierpliwy i
	wielkiego miłosierdzia, a współczujący w nieszczęściu. Kto wie, może się odwróci
	i wybaczy i zostawi po sobie błogosławieństwo na ofiarę i płynną obiate dla Pana
	Boga waszego. Uderzcie w trąbę na Syjonie, uświęćcie post, zwołajcie
	zgromadzenie, zbierzcie lud, poświęćcie zebranie, w jedno zbierzcie starców,
	zgromadźcie dzieci i niemowlęta. Niech wynijdzie oblubieniec z łożnicy swojej, a
	oblubienica ze swej komnaty. Pomiędzy przedsionkiem a ołtarzem płakać będą
	kapłani, słudzy Pańscy, i mówić będą: «Przepuść, Panie, przepuść ludowi Twemu, a
	dziedzictwa Twego na wzgardę nie wydawaj, by mieli nad nimi panować poganie.
	Czemuż mówią pomiędzy narodami: Gdzież jest ich Bóg?» Zapłonął Pan miłością ku
	ziemi swojej i przepuścił swemu ludowi. I odpowiedział Pan, i rzekł ludowi
	swemu: Oto ja ześle wam pszenicę i wino, i oliwę, a nasycicie się nimi. I nie
	wydam was na wzgardę poganom – mówi Pan wszechmogący.}

\proroctwo{Graduał (Ps 56:2; 56:4)}

\oremuss{Miserére mei, Deus, miserére mei: quóniam in te confídit ánima mea.

	\vv Misit de coelo, et liberávit me, dedit in oppróbrium conculcántes me.}
{Zmiłuj się nade mną, Boże, zmiłuj się nade mną, bo Tobie ufa dusza moja.

	\vv Sięgnął nieba i wybawił mnie, hańbą okrył tych, co mnie dręczą.}

\proroctwo{Traktus (Ps 102:10; 78:8-9)}

% \rubric{Na słowa \textit{Veni Sancte Spiritus} przyklęka sie.}

\oremuss{Dómine, non secúndum peccáta nostra, quæ fécimus nos: neque secúndum
iniquitátes nostras retríbuas nobis

Dómine, ne memíneris iniquitátum nostrarum antiquarum: cito antícipent nos
misericórdiæ tuæ, quia páuperes facti sumus nimis. {\footnotesize{Hic
	genuflectitur}} \medskip

\vv Adiuva nos, Deus, salutáris noster: et propter glóriam nóminis tui,
Dómine, libera nos: et propítius esto peccátis nostris, propter nomen
tuum.}{Panie, nie postępuj z nami według naszych grzechów ani nie odpłacaj
nam według win naszych.

Nie pamiętaj, Panie, dawnych nieprawości naszych. Niech rychło wyjdzie ku
nam miłosierdzie Twoje, bo jesteśmy bardzo nieszczęśliwi. {\footnotesize Tu
się przyklęka} \medskip

\vv Wspomóż nas, Boże, nasz Zbawicielu, i dla chwały imienia Twego wybaw
nas, Panie, i odpuść nam grzechy dla imienia Twego.}

\proroctwo{Ewangelia ( Mt 6:16-21)}

\oremuss{In illo témpore: Dixit Iesus discípulis suis: Cum ieiunátis, nolíte
	fíeri, sicut hypócritæ, tristes. Extérminant enim fácies suas, ut appáreant
	homínibus ieiunántes. Amen, dico vobis, quia recepérunt mercédem suam. Tu
	autem, cum ieiúnas, unge caput tuum, et fáciem tuam lava, ne videáris
	homínibus ieiúnans, sed Patri tuo, qui est in abscóndito: et Pater tuus, qui
	videt in abscóndito, reddet tibi. Nolíte thesaurizáre vobis thesáuros in
	terra: ubi ærúgo et tínea demólitur: et ubi fures effódiunt et furántur.
	Thesaurizáte autem vobis thesáuros in coelo: ubi neque ærúgo neque tínea
	demólitur; et ubi fures non effódiunt nec furántur. Ubi enim est thesáurus
	tuus, ibi est et cor tuum.}{Onego czasu: Mówił
	Jezus uczniom swoim: «Gdy pościcie, nie bądźcie smutni jako obłudnicy.
	Twarze bowiem swoje wyniszczają, aby okazali ludziom, że poszczą. Zaprawdę
	powiadam wam, że wzięli nagrodę swoją. Ale ty, gdy pościsz, namaść głowę
	swoją i obmyj oblicze swoje, abyś nie okazał ludziom, że pościsz, ale Ojcu
	twojemu, który jest w skrytości. A Ojciec twój, który widzi w skrytości,
	odda tobie. Nie skarbcie sobie skarbów na ziemi, gdzie rdza i mól psuje i
	gdzie złodzieje wykopują i kradną. Ale skarbcie sobie skarby w niebie, gdzie
	ani rdza zepsuje, ani mól stoczy i gdzie złodzieje nie wykopują ani kradną.
	Albowiem gdzie jest skarb twój, tam i serce twoje».}

\oremus{Offertorium (Ps 29:2-3)}{Exaltábo te, Dómine, quóniam suscepísti me, nec
	delectásti inimícos meos super me: Dómine, clamávi ad te, et sanásti
	me.}{Sławić Cię będę, Panie, bo mnie wybawiłeś i nie sprawiłeś ze mnie
	uciechy mym wrogom. Panie, do Ciebie wołałem, a Tyś mnie uzdrowił.}

\oremus{Sekreta}{Fac nos, quǽsumus, Dómine, his munéribus offeréndis
	conveniénter aptári: quibus ipsíus venerábilis sacraménti celebrámus
	exórdium.}{Spraw, prosimy Cię, Panie, abyśmy należycie się usposobili do
	ofiarowania Tobie tych darów, przez które obchodzimy początek czcigodnego
	misterium postu.}

\rubric{Prefacja Wielkopostna.}

\proroctwo{Pieśń na Komunie}

\gregorioscore{Melody/AD}

\bigskip 

\oremus{Communio (Ps 1:2,3)}{Qui meditábitur in lege Dómini die ac nocte, dabit
	fructum suum in témpore suo.}{Kto rozważa Prawo Pańskie dniem i nocą, ten wyda
	owoc w swym czasie.}

\oremus{Modlitwa po Komunii}{Percépta nobis, Dómine, præbeant sacraménta
	subsídium: ut tibi grata sint nostra ieiúnia, et nobis profíciant ad
	medélam.}{Niech nas wspomaga, Panie, przyjęty Sakrament, aby nasze posty
	były miłe Tobie, a nam przyniosły wyleczenie.}

\oremus{Modlitwa nad ludem}{Humiliáte cápita vestra Deo.\medskip

	Inclinántes se, Dómine, maiestáti tuæ, propitiátus inténde: ut, qui divíno
	múnere sunt refécti, coeléstibus semper nutriántur auxíliis.}{
	Pochylcie głowy wasze przed Bogiem. \medskip

	Wejrzyj łaskawie, o Panie, na wiernych, którzy chylą się przed Twoim majestatem,
	aby posileni Boskim darem zawsze byli podtrzymywani niebieską pomocą.}

\proroctwo{Pieśń na wyjście}

\gregorioscore{Melody/ABC}

\bigskip

\vfill

\centerline{\pgfornament[width=7cm]{82}}

\vfill


\end{document}







