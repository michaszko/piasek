%! TeX program = lualatex
%! TEX options = -synctex=1 -interaction=nonstopmode -file-line-error --shell-escape "%DOC%"
\documentclass[10pt,a5paper]{mszalik}

\begin{document}

\thispagestyle{empty}

\begin{center}
	\vspace*{0.5cm}


	\bfseries\scshape
	\huge NIESZPORY LUDOWE

    \vspace*{0.5cm}

    \huge I NIEDZIELA ADWENTU

    \vspace*{0.5cm}

	\vfill

	\begin{figure}[h]
		\centering
		\includegraphics[width=0.45\linewidth]{logo.png}
	\end{figure}

	\vfill

    \vspace*{0.5cm}

    {\large \textbf{30 listopada 2025}}

    \bigskip

    {\large Kolegiata Świętego Krzyża, Wrocław}

\end{center}

\newpage

\vspace*{2cm}

\vspace*{\fill}

\hrulefill

{\footnotesize

    Śpiewy i teksty liturgiczne pochodzą z:

	\begin{itemize}[leftmargin=*]
        % \item \textit{Antiphonale Monasticum} (1934),
        \item \textit{Liber antiphonarius} (1960),
		\item \textit{Liber Usualis} (1961),
        % \item \textit{Psalterium Monasticum} (1981),
        \item \textit{ks. Antoni Reginek -- Wybór pieśni nabożnych Franciszka Karpińskiego oraz psalmów w jego tłumaczeniu} (2001),
        \item \textit{ks. Jan Siedlecki -- Śpiewnik kościelny, wydanie XLI} (2021),
	\end{itemize}

	W niektórych przypadkach wykorzystano cyfrową wersję tekstów
	opublikowaną na portalach \textit{sbc.org.pl}, \textit{divinumofficium.com}, \textit{missalemeum.com} i \textit{spiewnik.katolicy.net}.

    \medskip

    Zapis gabc neumów: Piotr Matuszak; Andrew Hinkley, Frado, mortonsc \textit{gregobase.selapa.net}.

	% \medskip

    % Duszpasterstwo Wiernych Tradycji Łacińskiej w Archidiecezji Wrocławskiej jest duszpasterstwem
    % erygowanym przez J.E. ks. abp. Józefa Kupnego, Metropolitę Wrocławskiego.

	\bigskip

	Opracowanie: Mateusz Burszewski, Piotr Matuszak\\
	Skład i łamanie: Michał Siemaszko, Piotr Matuszak\\

	\texttt{tradycja.archidiecezja.wroc.pl}
}


%%%%%%%%%%%%%%%%%%%%%%%%%%%%%%%%%%%%%%%%%%%%%%%%%%%%%%%%%%%%%%%%%%%%%%%%%%

\newpage

\fancyhead[CE]{NIESZPORY LUDOWE}
\fancyhead[CO]{I NIEDZIELA ADWENTU}

\section{NIESZPORY LUDOWE}

\rubric{ Kapłan i najbliższa asysta przyklękają przed ołtarzem. Klękają i modlą się w ciszy po czym wstają i przechodzą do sedili.}


\gresetinitiallines{0}
\gregorioscore{Melody/boze-wejrzyj.gabc}
\gresetinitiallines{1}

\medskip

Chwała bądź Bogu w Trójcy \textbf{je}dynemu: \redcol{*} Ojcu, Synowi, Ducho\textit{wi} Świętemu. \\
Jak od początku była, \textbf{tak} i ninie, \redcol{*} I na wiek wieków niechaj \textit{zaw}sze słynie. \\
Alleluja.

\gregorioscore{Melody/an--in_illa_die--solesmes_1961.1.gabc}

\proroctwomedskip{Psalm 109}

\begin{lilypond}
\relative c' {
    \key f \major
    \clef treble
    \override Staff.TimeSignature.stencil = ##f
    \override Staff.NoteHead.style = #'altdefault
    \cadenzaOn
    a'\breve bes8 a8 \parenthesize a8 g8 a4 \bar "|" \break
    g\breve a8 \parenthesize a8 g8 f8 bes8([a8]) g4 \bar "||"
    \cadenzaOff
}
\addlyrics {
    \override LyricText.font-size = #-1
    Rzekł_Pan_do_Pana_ \markup { \bold me } -- go łas -- ka -- wym,
    Swym_głosem:_siądź_mi \markup { \bold przy } _ bo -- ku pra -- wym.
}
\end{lilypond}

% 1. Rzekł Pan do Pana \textbf{me}go łaskawym, \redcol{*} Swym głosem: siądź mi \textbf{przy} boku prawym.\\
2. Aż Twoje wszystkie \textbf{zu}chwałe wrogi, \redcol{*} Dam za podnóżek \textbf{pod} Twoje nogi.\\
3. Berło Twej mocy \textbf{wy}dam z Syjonu, \redcol{*} świat cały padnie \textbf{u} Twego tronu.\\
4. A Ty używać \textbf{bę}dziesz praw swoich, \redcol{*} wśród nieprzyjaciół \textbf{pa}nując Twoich.\\
5. Skoro rozpoczniesz \textbf{swe} panowanie, \redcol{*} przy Tobie lud Twój w \textbf{jas}ności stanie.\\
6. Nim jeszcze zorzy \textbf{świe}cić kazałem, \redcol{*} Ciebie z wnętrzności \textbf{swo}ich wydałem.\\
7. Pan to poprzysiągł, \textbf{Jego} zaś mowa \redcol{*} danego nigdy \textbf{nie} cofnie słowa.\\
8. Ty jesteś kapłan \textbf{do} końca wieka, \redcol{*} według obrządku \textbf{Mel}chizedeka.\\
9. Przy Twej prawicy \textbf{Pan} jest nad pany, \redcol{*} w dzień gniewu swego \textbf{ze}trze tyrany.\\
10. Sąd swój rozciągnie \textbf{po} całym świecie \redcol{*} i nieposłuszne \textbf{na}rody zgniecie.\\
11. Pyszną na ziemi \textbf{gło}wę poniży, \redcol{*} która Mu Jego \textbf{chwa}ły ubliży.\\
12. Z mętnej po drodze \textbf{pić} będzie rzeki, \redcol{*} dlatego głowę \textbf{wznie}sie na wieki.\\
13. Chwała bądź Bogu w Trójcy \textbf{Je}dynemu, \redcol{*} Ojcu, Synowi, Du\textbf{cho}wi Świętemu.\\
14. Jak od początku była, \textbf{tak} i ninie, \redcol{*} I na wiek wieków \textbf{nie}chaj zawsze słynie.\\

\gregorioscore{Melody/an--in_illa_die--solesmes_1961.1.gabc}

\medskip

\gregorioscore{Melody/an--jucundare_filia--solesmes_1961.1.gabc}

\proroctwomedskip{Psalm 110}

\begin{lilypond}
\relative c' {
    \key f \major
    \clef treble
    \override Staff.TimeSignature.stencil = ##f
    \override Staff.NoteHead.style = #'altdefault
    \cadenzaOn
    f\breve d8 d4 c4 \bar "|" \break
    f\breve g8 a8 g4 f4 \bar "||"
    \cadenzaOff
}
\addlyrics {
    \override LyricText.font-size = #-1
    Całym_Cię_sercem_chwalić_bę -- \markup { \bold dą, } Pa -- nie,
    gdzie_rada_mędrców,_gdzie_ca -- \markup { \bold ły } zbór sta -- nie.
}
\end{lilypond}

% 1. Całym Cię sercem chwalić bę\textbf{dę}, Panie, gdzie rada mędrców, gdzie ca\textbf{ły} zbór stanie.
2. Wielkość niezmierna dzieł Boga \textbf{Ja}kuba, \redcol{*} Którymi władnie, jak Mu \textbf{się} podoba.\\
3. Co pocznie, wszystko pełne jest \textbf{zac}ności, \redcol{*} Pełne uwielbień. Wiek spra\textbf{wie}dliwości.\\
4. Jego nie przetrwa żaden, a \textbf{na} ziemi, \redcol{*} Wieczną pamiątkę sprawi \textbf{cu}dy Swymi.\\
5. Pan dobrotliwy, Pan to mi\textbf{ło}sierny, \redcol{*} Karmi i hojnie boga\textbf{ci} lud wierny.\\
6. Stateczny w słowie, co raz po\textbf{sta}nowił, \redcol{*} Wiecznymi czasy tego \textbf{nie} odmówił.\\
7. Moc Swą okazał, gdy wygnał \textbf{po}gany, \redcol{*} A ten kraj synom Izra\textbf{e}la dany.\\
8. Sprawy rąk Jego prawdą miar\textbf{ko}wane, \redcol{*} A zaś słusznością prawa \textbf{pro}stowane.\\
9. Te się na wieki żadnych lat \textbf{nie} boją, \redcol{*} Bo na słuszności i na \textbf{praw}dzie stoją.\\
10. Pan sługi Swoje z niewoli \textbf{wy}bawił, \redcol{*} I dla nich wieczny testa\textbf{ment} zostawił.\\
11. Imię ma straszne i pełne \textbf{świę}tości, \redcol{*} A bojaźń Boża począt\textbf{kiem} mądrości.\\
12. Ci, co z niej biorą wzór życia \textbf{na} ziemi, \redcol{*} Będą od wszystkich wiecznie \textbf{chwa}lonymi.\\
13. Chwała bądź Bogu w Trójcy Je\textbf{dy}nemu, \redcol{*} Ojcu, Synowi, Ducho\textbf{wi} Świętemu.\\
14. Jak od początku była, tak \textbf{i} ninie, \redcol{*} I na wiek wieków niechaj \textbf{zaw}sze słynie.\\

\gregorioscore{Melody/an--jucundare_filia--solesmes_1961.1.gabc}

\gregorioscore{Melody/an--ecce_dominus_veniet--solesmes_1961.1.gabc}

\proroctwomedskip{Psalm 111}

\begin{lilypond}
\relative c' {
    \key f \major
    \clef treble
    \override Staff.TimeSignature.stencil = ##f
    \override Staff.NoteHead.style = #'altdefault
    \cadenzaOn
    a'\breve g8 a8 bes4 bes4 \bar "|" \break
    a\breve bes8 a8 g4 f4 \bar "||"
    \cadenzaOff
}
\addlyrics {
    \override LyricText.font-size = #-1
    Szczęśliwy_i  \markup { \bold nie } zna kaź -- ni,
    kto_w_Pańskiej_ży -- \markup { \bold je } bo -- jaż -- ni.
}
\end{lilypond}
% 1. Szczęśliwy nie zna kaźni, \redcol{*} Kto w Pańskiej żyje bojaźni.\\
2. Najmilsza je\textbf{mu} jest droga, \redcol{*} iść według przy\textbf{ka}zań Boga.\\
3. Krew jego zac\textbf{na} na ziemi, \redcol{*} porówna zaw\textbf{żdy} z możnymi.\\
4. Ród się cnotli\textbf{wych} rozpieni, \redcol{*} i będą bło\textbf{go}sławieni.\\
5. Dom jego bę\textbf{dzie} obfity, \redcol{*} w zbiory i trwa\textbf{łe} zaszczyty.\\
6. A cześć poczci\textbf{we}go człeka, \redcol{*} pójdzie od wie\textbf{ka} do wieka.\\
7. Niechaj noc zać\textbf{mi} mgły swymi, \redcol{*} światło jest nad \textbf{cno}tliwymi.\\
8. Zawsze im Pan \textbf{jest} życzliwy, \redcol{*} litosny i \textbf{spra}wiedliwy.\\
9. Człek miły peł\textbf{en} wesela, \redcol{*} gdy drugiemu z swe\textbf{go} udziela.\\
10. W słowie się swo\textbf{im} tak rządzi, \redcol{*} że, co wyrze\textbf{cze}, nie zbłądzi.\\
11. Ludzka pamięć \textbf{jego} sprawy \redcol{*} uwieczni: „Był \textbf{to} mąż prawy”.\\
12. I zły go ję\textbf{zyk} nie trwoży, \redcol{*} bezpieczny w nad\textbf{zie}i Bożej.\\
13. Stateczność je\textbf{go} cnej duszy, \redcol{*} żadną się rze\textbf{czą} nie wzruszy.\\
14. Wydoła złej \textbf{chwi}li snadnie, \redcol{*} aż nieprzyja\textbf{ciel} upadnie.\\
15. Rozsypał swo\textbf{je} szczodroty, \redcol{*} na wdowy, bied\textbf{ne} sieroty.\\
16. Przeto uczyn\textbf{no}ścią słynie, \redcol{*} i chwała je\textbf{go} nie zginie.\\
17. Zły, na to pa\textbf{trząc}, boleje, \redcol{*} zgrzyta, z zazdro\textbf{ści} sinieje.\\
18. Taki, co w gło\textbf{wie} uradzi, \redcol{*} do skutku nie \textbf{do}prowadzi.\\
19. Chwała Ojcu \textbf{i} Synowi, \redcol{*} oraz Święte\textbf{mu} Duchowi.\\
20. Jak na począ\textbf{tku}, tak ninie, \redcol{*} i na wieki \textbf{nie}chaj słynie.\\

\gregorioscore{Melody/an--ecce_dominus_veniet--solesmes_1961.1.gabc}

\medskip

\gregorioscore{Melody/an--omnes_sitientes--solesmes_1961.1.gabc}

\proroctwo{Psalm 112}

\begin{lilypond}
\relative c' {
    \key g \major
    \clef treble
    \override Staff.TimeSignature.stencil = ##f
    \override Staff.NoteHead.style = #'altdefault
    \cadenzaOn
    g'8 d\breve e8 fis8 g8([fis8]) e8 d8([c8]) b4 \bar "|" \break
    a'\breve b8 c8 b8 a4 g4 \bar "||"
    \cadenzaOff
}
\addlyrics {
    \override LyricText.font-size = #-1
    Chwal -- cie,_o_dziatki,  \markup { \bold Naj } -- wyż -- sze -- go Pa -- na.
    Niech_Mu_jednemu_cześć \markup { \bold bę} -- dzie śpie -- wa -- na.
}
\end{lilypond}

% 1. Chwalcie o dziatki, Najwyższego Pana, \redcol{*} niech Mu jednemu cześć będzie śpiewana.\\
2. Niech Imię Pańskie \textbf{prze}błogosławione \redcol{*} po wszystkie wieki bę\textbf{dzie} uwielbione.\\
3. Gdzie wschodzi słońce \textbf{i} kędy zapada, \redcol{*} niechaj świat Boską chwa\textbf{łę} opowiada.\\
4. Pan ma narody \textbf{wszyst}kie pod nogami. \redcol{*} Jego się chwała wzno\textbf{si} nad gwiazdami.\\
5. Kto się wżdy z Panem \textbf{tym} porówna, który \redcol{*} wysoko siedząc, z swej \textbf{nie}bieskiej góry.\\
6. I co na niebie, \textbf{i} co jest na ziemi, \redcol{*} oczyma widzi nie\textbf{u}chronionymi.\\
7. On ubogiego \textbf{z nę}dzy wyprowadzi \redcol{*} i z książętami na \textbf{ła}wie posadzi.\\
8. On niesie radość \textbf{dla} niepłodnej matki, \redcol{*} miłe w jej domu roz\textbf{mna}żając dziatki.\\
9. Chwała bądź Bogu \textbf{w Trój}cy jedynemu, \redcol{*} Ojcu, Synowi, Du\textbf{cho}wi Świętemu.\\
10. Jak od początku \textbf{by}ła, tak i ninie, \redcol{*} I na wiek wieków nie\textbf{chaj} zawsze słynie.\\

\medskip

\gregorioscore{Melody/an--omnes_sitientes--solesmes_1961.1.gabc}

\gregorioscore{Melody/an--ecce_veniet_propheta--solesmes_1961.1.gabc}

\proroctwomedskip{Psalm 113}

\begin{lilypond}
\relative c' {
    \key ees \major
    \clef treble
    \override Staff.TimeSignature.stencil = ##f
    \override Staff.NoteHead.style = #'altdefault
    \cadenzaOn
    bes'8 aes8 g\breve aes8 g8 f4 ees4 \bar "|" \break
    f\breve g8([f8]) ees8 d4 c4 \bar "||"
    \cadenzaOff
}
\addlyrics {
    \override LyricText.font-size = #-1
    Kie -- dy z_Egiptu_Izra --  \markup { \bold el } wy -- cho -- dził,
    dom_się_Jakuba_z_więzów  \markup { \bold o } -- swo -- bo -- dził.
}
\end{lilypond}

% 1. Kiedy z Egiptu Izra\textbf{el} wychodził, \redcol{*} dom się Jakuba z więzów \textbf{o}swobodził.\\
2. Wielka tam, Panie! łaska \textbf{Two}ja była, \redcol{*} i niepodobna ku wie\textbf{rze}niu siła.\\
3. Prąd morza razem na pół \textbf{się} rozsadził, \redcol{*} Jordan w tył wody swoje \textbf{od}prowadził;\\
4. Góry wzniesione, jak ca\textbf{py}; a skały, \redcol{*} jako wesoła jagnię\textbf{ta} skakały.\\
5. Cóż to ci morze? kto cię \textbf{tak} rozdziera? \redcol{*} kto twe Jordanie w tył wo\textbf{dy} odpiera?\\
6. Góry! czemuście, jak ca\textbf{py} skakały? \redcol{*} i wy dla czego, jak jag\textbf{nię}ta, skały?\\
7. Od twarzy Pana zosta\textbf{ły} wzruszone, \redcol{*} Ziemia i morze, i gó\textbf{ry} wzniesione!\\
8. On zdroje z głazów, a \textbf{z twar}dej opoki \redcol{*} on wyprowadził swą rę\textbf{ką} potoki.\\
9. Nie nam, o Panie! nie nam, \textbf{a}le Twemu \redcol{*} daj Imieniowi chwałę \textbf{Naj}świętszemu.\\
10. Niechaj wiadome i sła\textbf{wio}ne wszędzie, \redcol{*} Twe miłosierdzie, Twoja \textbf{praw}da będzie.\\
11. Niechaj pohańce sprośni \textbf{nie} pytają: \redcol{*} „gdzież teraz ich Bóg, które\textbf{mu} dufają?”\\
12. Nasz Bóg na niebie, cokol\textbf{wiek} zamyśli, \redcol{*} wszystko się musi stać po \textbf{Je}go myśli.\\
13. A ich bałwany ze sreb\textbf{ra}, ze złota, \redcol{*} nic nie są, tylko ludzkich \textbf{rąk} robota:\\
14. Gębą nie mówią, okiem \textbf{nie} patrzają, \redcol{*} uchem nie słyszą, i wo\textbf{ni} nie mają.\\
15. Ręką nie ścisną, nie pos\textbf{tą}pią nogą, \redcol{*} gardłem żadnego głosu \textbf{dadź} nie mogą:\\
16. Bodaj tak i ci, którzy \textbf{je} działają, \redcol{*} a owszem, którzy w nich na\textbf{dzie}ję mają.\\
17. Izrael w Panu swe na\textbf{dzie}je stawił, \redcol{*} a on go wszelkiey przygo\textbf{dy} pozbawił:\\
18. Niech się na Pana spuści \textbf{dom} Aarona, \redcol{*} jawna mu jego dobroć \textbf{i} obrona.\\
19. Niech mu dufają, którzy \textbf{się} Go boją, \redcol{*} bo krom wątpienia w łasce \textbf{Je}go stoją;\\
20. Pan o nas pomni, Pan nas błogo\textbf{sła}wił; \redcol{*} dom Izraela, Aa\textbf{ro}na, wsławił.\\
21. Każdemu łaskaw poboż\textbf{ne}mu człeku, \redcol{*} tak w małym, jako i do\textbf{ro}słym wieku;\\
22. Niechże swą łaskę pomno\textbf{ży} nad wami, \redcol{*} nad wami, i nad waszy\textbf{mi} dziatkami.\\
23. Pan wam na wieki będzie \textbf{bło}gosławił; \redcol{*} Ten, który niebo i zie\textbf{mię} wystawił;\\
24. Niebo wysokie Jego \textbf{jest} mieszkanie, \redcol{*} a ziemię ludziom oddał \textbf{w u}żywanie.\\
25. Nie martwi, Panie! będą \textbf{Cię} chwalili, \redcol{*} ani ci, którzy pod zie\textbf{mię} wstąpili,\\
26. Ale my, którzy na świe\textbf{cie} żyjemy, \redcol{*} wiecznymi czasy sławić \textbf{Cię} będziemy.\\
27. Chwała bądź Bogu w Trójcy \textbf{je}dynemu, \redcol{*} Ojcu, Synowi, Ducho\textbf{wi} Świętemu.\\
28. Jak od początku była \textbf{tak} i ninie, \redcol{*} i na wiek wieków niechaj \textbf{zaw}sze słynie.\\

\gregorioscore{Melody/an--ecce_veniet_propheta--solesmes_1961.1.gabc}

\proroctwomedskip{Kapitulum}

\vv Fratres: Hora est jam nos de somno s\'{u}rgere:
nunc enim pr\'{o}pior est nostra salus, quam cum cred\'{i}dimus.

\medskip

% \rr Deo grátias.
\gresetinitiallines{0}
\gregorioscore{Melody/deo-gratias.gabc}
\gresetinitiallines{1}

\newpage

\proroctwomedskip{Hymn}

\gregorioscore{Melody/hy--creator_alme_siderum--solesmes_1961.gabc}

\bigskip

% \vv Dirigátur, Dómine, orátio mea.

\gresetinitiallines{0}
\gregorioscore{Melody/rorate-caeli-desuper.gabc}
\gresetinitiallines{1}

\rr Aperiátur terra, et gérminet Salvató\textbf{rem}.

\proroctwomedskip{Magnificat}

\gregorioscore{Melody/an--ne_timeas_maria--solesmes_1934.gabc}

\medskip

\begin{lilypond}
\relative c' {
    \key f \major
    \clef treble
    \override Staff.TimeSignature.stencil = ##f
    \override Staff.NoteHead.style = #'altdefault
    \cadenzaOn
    f\breve g8 g8 a8([g8]) f4 \bar "|" \break
    f\breve e8 f8 d4 c4 \bar "|" \break
    a'\breve g8 a8 bes4 bes4 \bar "|" \break
    a\breve g8 f8 g4 f4 \bar "||"
    \cadenzaOff
}
\addlyrics {
    \override LyricText.font-size = #-1
    Uwielbiaj,_duszo_moja,_sławę \markup { \bold Pa } -- na me -- go,
    chwal_Boga_Stworzyciela_tak_bar -- \markup { \bold dzo } dob -- re -- go.
    Bóg_mój,_zbawienie_moje,_jedy -- \markup { \bold na } o -- tu -- cha,
    Bóg_mi_rozkoszą_serca_i_we -- \markup { \bold se } -- lem du -- cha.
}
\end{lilypond}

% 1. Uwielbiaj, \ding{64} duszo moja, sławę \textbf{Pa}na mego, \redcol{*} Chwal Boga Stworzyciela tak bar\textbf{dzo} dobrego.\\
% 2. Bóg mój, zbawienie moje, jedy\textbf{na} otucha, \redcol{*} Bóg mi rozkoszą serca i we\textbf{se}lem ducha.\\
3. Bo mile przyjąć raczył swej słu\textbf{gi} pokorę, \redcol{*} łaskawym okiem wejrzał na Da\textbf{wi}da córę.\\
4. Przeto wszystkie narody, co zie\textbf{mię} osiędą, \redcol{*} Odtąd błogosławioną mnie na\textbf{zy}wać będą.\\
5. Bo wielkimi darami uczczo\textbf{nam} od Tego, \redcol{*} którego moc przedziwna, święte \textbf{I}mię Jego.\\
6. Którzy się Pana boją, szczęśli\textbf{wi} na wieki, \redcol{*} bo z nimi miłosierdzie z rodu \textbf{w ród} daleki.\\

\newpage 

7. Na cały świat pokazał moc swych \textbf{ra}mion świętych, \redcol{*} rozproszył dumne myśli głów py\textbf{chą} nadętych.\\
8. Wyniosłych złożył z tronu, znikczem\textbf{nił} wielmożne, \redcol{*} wywyższył, uwielmożnił w poko\textbf{rę} zamożne.\\
9. Głodnych nasycił, hojnie i w dob\textbf{ra} spanoszył, \redcol{*} bogaczów z torbą puścił i nędz\textbf{nie} rozproszył.\\
10. Przyjął do łaski sługę Izra\textbf{e}la cnego, \redcol{*} wspomniał nań, użyczył mu miło\textbf{sier}dzia swego.\\
11. Wypełnił, co był przyrzekł niegdyś \textbf{oj}com naszym: \redcol{*} Abrahamowi z potomstwem, jego \textbf{wiecz}nym czasem.\\
{\footnotesize\color{red}{\textbf{Improwizacja organowa do czasu, aż turyfer powróci od okadzenia ludu.}}}\\
12. Wszyscy śpiewajmy Bogu w Trójcy \textbf{Je}dynemu, \redcol{*} Chwała Ojcu, Synowi, Ducho\textbf{wi} Świętemu.\\
13. Jak była na początku, tak zaw\textbf{sze} niech będzie, \redcol{*} teraz i na wiek wieków niechaj \textbf{sły}nie wszędzie.\\

\gregorioscore{Melody/an--ne_timeas_maria--solesmes_1934.gabc}

\medskip

\proroctwomedskip{Oracja}

\vv Dóminus vobíscum.

\rr Et cum spíritu tuo.

\vv Oremus. Excita, quǽsumus, Dómine, poténtiam tuam, et veni: ut ab imminéntibus peccatórum nostrórum perículis,
te mereámur protegénte éripi, te liberánte salvári:
Qui vivis et regnas cum Deo Patre, in unitáte Spíritus Sancti, Deus, per ómnia sǽcula sæculórum. 

\rr Amen.

\proroctwomedskip{Zakończenie}

\vv Dóminus vobíscum.

\rr Et cum spíritu tuo.

% \vv Benedicamus Domino.

% \rr Deo gratias.

\gresetinitiallines{0}
\gregorioscore{Melody/or--benedicamus_domino_sundays_of_advent_and_lent--solesmes_1961.1.gabc}
\gresetinitiallines{1}

\medskip

\vv Fidélium ánimæ per misericórdiam Dei requiéscant in pace.

\rr Amen.

\proroctwomedskip{Antyfona}

\gregorioscore{Melody/an--alma_redemptoris--solesmes.1.gabc}

\end{document}
